\chapter*{ЗАГАЛЬНА ХАРАКТЕРИСТИКА РОБОТИ}

\newcommand{\actuality}{\textbf{Актуальність теми дослідження.}}
\newcommand{\relationship}{\textbf{Зв’язок роботи з науковими програмами, планами, темами.}}
\newcommand{\aim}{\textbf{Мета і задачі дослідження.}}
\newcommand{\objectofstudy}{\textit{Об’єктом дослідження}}
\newcommand{\subjectofstudy}{\textit{Предметом дослідження}}
\newcommand{\methodofstudy}{\textit{Методи дослідження.}}
\newcommand{\defpositions}{\textbf{Основні положення для захисту}}
\newcommand{\novelty}{\textbf{Наукова новизна одержаних результатів.}}
\newcommand{\influence}{\textbf{Практичне значення одержаних результатів.}}
\newcommand{\probation}{\textbf{Апробація результатів дисертації.}}
\newcommand{\contribution}{\textbf{Особистий внесок здобувача.}}
\newcommand{\publications}{\textbf{Публікації.}}

{\actuality} Значні труднощі в побудові коректних моделей фізико-математичних, біологічних, хімічних і інших явищ і
процесів полягають в недостатній повноті і достовірності інформації про ці процеси. Щоб розглядати їх у глобальній і
локальній перспективі необхідно апроксимувати значення вимірів у відомих точках (моментах) або будувати
інтегро-диференціальні (інтегральні) математичні моделі, і встановлювати їхню адекватність. Не викликає сумніву те,
що подібні задачі в класичному розумінні будуть некоректними, та такими, що мають безліч розв’язків. Дослідження
їхньої стійкості, існування розв’язків (фізичних, математичних, машинних) є також необхідним.

Відомо, що процеси тепло-масо-переносу відіграють значну роль при проектуванні та оптимізації деталей машин та
механізмів, елементів конструкцій ядерних реакторів, дослідженні та аналізу теплового забруднення ґрунтів тощо.
Велика кількість цих процесів може бути описана лінійними диференціальними, інтегральними та інтегро-диференціальними
моделями. Вони дозволяють отримувати розв’язки як прямих так і обернених задач теплопровідності з доволі високої
точністю. Їм присвячені роботи О.М. Аліфанова, М.М. Бєляєва,О.І.Буковської, В.І. Буренко, Г.П. Донцової, О.І.
Каліткіна, Л.О. Коздоби, С.С. Кутателадзе, О.В. Ликова та інших. Однак, практичні дослідження доводять, що на
теперішній час більшість актуальних задач теплопровідності описується нелінійними моделями. Більш точне врахування
властивостей матеріалів, аналіз роботи елементів механізмів у пограничних теплових режимах а також при швидких
змінах температур, оптимізація роботи цих елементів по теплових показниках – розв’язок всіх цих задач потребує
використання нелінійних математичних моделей. Методи розв’язків таких описані в чисельних працях, зокрема у працях
М.О. Авдоніна, О.М. Айзена, Г.І. Аксенова, Г.П. Бойкова, Б.М. Будака, О.В. Кавадерова, Л.О. Коздоби, Л.И. Кудряшова,
О.В. Ликова, Д.В. Маріна, О.Б. Москальова,  В.Л. Рвачова, В.В. Саломатова, О.В. Фурмана тощо.
Серед нелінійних задач теплопровідності мають велику практичну значущість задачі керування тепловими системами,
задачі спостереження та відновлення теплових характеристик матеріалів по відомих теплових полях. Очевидно, що ці
задачі є некоректними у класичному розумінні. Одним з методів розв’язання некоректних  задач є метод регулярізації,
запропонований А.М. Тихоновим.

Серед праць, присвячених цій тематиці, позначимо  роботи О.В. Гончарського, В.В.
Степанова, В.Д. Кальнера, Б.В.Гласко, О.О. Самарського, В.Я  Арсеніна. Розробці методів моделювання лінійних прямих
та обернених задач динаміки систем з розподіленими параметрами присвячені роботи В.А. Стояна, В.В. Скопецького,
Ю.Г. Кривоноса,Т.Ю. Благовещенської, В.О. Богаєнко. Загальні питання керування лінійними системами освітлені в
працях Б.М. Бублика, М.Ф. Кириченко, С.І. Ляшко та інших.

Розв’язку прямих та обернених задач теплопровідності в умовах неповноти інформації і присвячена дана дисертаційна
робота. Основна ідея полягає у розширенні методики В.А. Стояна, розвинутої у працях Т.Ю. Благовещенської,
В.О. Богаєнко на задачі нелінійні задачі теплопровідності які описуються математичними моделями з неповними даними.

{\relationship} Робота виконана у відділі “Математичних систем моделювання проблем екології і енергетики” Інституту
кібернетики імені В.М. Глушкова НАН України за період 2003 – 2006 рр. Частина наукових та практичних результатів
отримана у рамках виконання:
\begin{itemize}
  \item проекту 01.07/0005 «Розробка математичних методів та інформаційних технологій моделювання прямих та обернених
  задач динаміки систем з розподіленими параметрами» Міносвіти і науки України (2001 - 2005рр.).
  \item держбюджетної теми В.Ф.К.175.10 «Розробка високоефективних інформаційних технологій прогнозу та розпізнавання
  ситуацій в системах прийняття рішень. Нові інформаційні технології (математичні моделі, алгоритми та програми)
  аналізу та синтезу фізико-механічних полів» НАН України (2002 – 2005 рр.).
\end{itemize}

{\aim} Метою даної роботи є  розробка методів побудови та дослідження загальних розв’язків (або середньоквадратичних
 наближень до них) нелінійних прямих та обернених задач теплопровідності.

Для цього розв’язані наступні задачі:
\begin{enumerate}
  \item Розширена область застосування  методики моделювання лінійних систем з розподіленими параметрами за допомогою
  функції Гріна на системи  з нелінійністю, що визначається малим параметром;
  \item Знайдені множини функцій, які моделюють стан теплової системи та виконаний перехід від диференціальною
  моделі теплових процесів до інтегральної або функціональної моделі;
  \item Розроблені алгоритми побудови загальних розв’язків (або середньоквадратичних наближень до них) прямих
  та обернених (задача спостереження) задач  теплопровідності для нелінійних систем з малим параметром в
  умовах структурної неповноти інформації;
  \item Встановлені оцінки точності розв’язків у необмежених, частково обмежених та повністю обмежених  областях;
  \item Проведене комп’ютерне моделювання динаміки нелінійних систем з розподіленими параметрами та виконаний
  аналіз його результатів.
\end{enumerate}

{\objectofstudy} є система з розподіленими параметрами, задана нелінійним в малому диференційним рівнянням в частинних
похідних та неповно визначеними початково-крайовими умовами.

{\subjectofstudy} є функція стану нелінійної в малому розподіленої системи з неповно визначеними початково-крайовими
умовами, множини фіктивних зовнішньо-динамічних збурень, моделюючих вплив початково-крайових умов, функція керування
нелінійною системою, похибка вказаного моделювання зовнішньо-динамічних збурень, похибка моделювання нелінійних збурень диференційного оператора.

{\methodofstudy} В роботі використано методи псевдоінверсні методи обернення систем інтегральних та функціональних
рівнянь, методи моделювання збурень нелінійних в малому диференційних операторів, чисельні методи розв’язку задач з
розподіленими параметрами.

{\novelty}
\begin{enumerate}
  \item Поширена загальна методика моделювання лінійних систем з розподіленими параметрами на нелінійні системи
  з малим параметром.
  \item Знайдені множини функцій, що середньоквадратично наближаються до \\ розв’язку прямих та обернених задач
  моделювання динаміки нелінійних систем, які описують теплові процеси.
  \item Побудовані множини фіктивних функцій, якими моделюється стан оточуючого середовища.
  \item Досліджені умови точності отриманих розв’язків
  \item Розроблені алгоритми оптимізації нелінійних моделей, що дозволяють за рахунок вибору точок дискретизації
  системи знизити похибку моделювання.
\end{enumerate}

{\influence} Розроблені методики та алгоритми дозволяють проводити моделювання динаміки нелінійних задач
теплопровідності в умовах неповноти інформації та отримувати множини функцій що середньоквадратично наближуються
до розв’язків прямих  та обернених задач теплопровідності. Знайдені методи дозволяють виконати оцінку точності
результатів моделювання. Отримані методики можуть бути використані в задачах моделювання процесів переносу в
акустиці, гідродинаміці, фільтрації та дифузії, в задачах оцінки теплового забруднення ґрунтів.


{\probation}
Основні результати дисертаційної роботи доповідались та обговорювались на міжнародних конференціях,
«Сучасні проблеми прикладної математики та інформатики» (Львів, 2006), «Одинадцята міжнародна наукова конференція
імені академіка М. Кравчука» (Київ, 2006), «Університет місту: XIII регіональна науково-технічна конференція»
(Маріуполь, 2006), «Сучасні проблеми математики та її застосування в природничих науках та інформаційних технологіях»
(Харків, 2007)наукових семінарах в Інституті кібернетики імені В.М. Глушкова НАН України (2003-2007).

{\contribution} Всі результати, які становлять суть  дисертаційних досліджень, отримані автором особисто та в співпраці
з науковим керівником. У публікації \cite{kharkiv-2007}, написаній у співавторстві з професором В.В. Скопецьким, останньому
належить постановка задачі та участь в обговоренні результатів.

{\publications} За темою дисертації опубліковано 7 наукових праць, з них три статті у вітчизняних фахових виданнях,
4 – у збірниках доповідей міжнародних та всеукраїнських конференцій.



\textbf{Структура та обсяг дисертаційної роботи.} Дисертаційна робота складається із вступу, чотирьох розділів,
висновків, бібліографічного списку з  найменувань.

Загальний обсяг роботи
з~
та~. Перелік посилань містить

\vspace{0pt plus1fill}
