\chapter{Чисельне моделювання розв’язків задач динаміки нелінійних систем} \label{chapt4}

В розділі нами виконано чисельне моделювання розв’язків нелінійних задач теплопровідності. Описана архітектура
програмного забезпечення, розробленого для моделювання, представлені чисельні розв’язки задач.

\section{Бібліотека математичних класів для проведення динаміки розподілених систем} \label{sect4_1}

\textit{Вибір мови програмування}. Для розв'язку поставлених задач ми будемо використовувати мову програмування Python
через декілька причин:
\begin{itemize}
    \item наявність великої екосистеми для наукових розрахунків (NumPy, SciPy, SymPy, Matplotlib, Pandas);
    \item гарна інтеграція з бібліотеками C/C++ (через Cython, Numba, F2PY), що дозволяє оптимізувати критичні ділянки;
    \item легкість прототипування, що є важливим для тестування чисельних методів;
    \item можливість вбудовувати код як у вебзастосунок (через Django, Flask, FastAPI), так і в десктопні застосунки GUI (PyQt, Tkinter).
\end{itemize}

\textit{Математична бібліотека розв'язку систем лінійних алгебраїчних рівнянь}. Для мови програмування Python існує декілька
бібліотек:
\begin{itemize}
    \item \href{https://docs.scipy.org/doc/scipy/reference/linalg.html#module-scipy.linalg}{Linear Algebra (scipy.linalg)} -- для точних методів (LU, QR, SVD);
    \href{https://docs.scipy.org/doc/scipy/reference/sparse.linalg.html}{Sparse linear algebra (scipy.sparse.linalg)}  -- для розріджених систем (методи Крилова, GMRES, BiCGSTAB);
    Для великих задач — \href{https://petsc.org/release/}{PETSc, the Portable, Extensible Toolkit for Scientific Computation} чи
    \href{https://pypi.org/project/pyamg/}{PyAMG, library of Algebraic Multigrid (AMG)} алгебраїчний багатосеточний метод.
\end{itemize}

\textit{Методи розв'язку систем лінійних алгебраїчних рівнянь}. За крітерієм точності методи розв'язку систем лінійних алгебраїчних рівнянь
ми можемо поділити на
\begin{itemize}
    \item точні методи Lower–Upper (LU)-розкладання~\cite{Lay-Linear-Algebra-and-its-Applications},
    QR-розкладання~\cite{Holmes-Introduction-to-cientific-Computing-and-Data-Analysis}, використання сингулярного розклад матриці~\cite{Applications-of-the-singular-value-decomposition-in-dynamics}
    Singular Value Decomposition (SVD) -- підходять для малих та середніх задач, де матриця щільна та розміри до ~$10^4$;
    \item наближені ітераційні методи:
    \begin{itemize}
        \item метод сполучених градієнтів~\cite{Hestenes-Methods-of-conjugate-gradients}, (Conjugate gradient, CG);
        \item узагальнений метод мінімального залишку~\cite{Qinmeng-Zou-GMRES-algorithms-over-35-years}, (Generalized Minimal Residual Method, GMRES);
        \item бі-сполучені градієнти зі стабілізацією~\cite{Saad-Iterative-Methods-for-Sparse-Linear-Systems-2nd-ed},
        (Bi-conjugate gradients with stabilization, BiCGSTAB);
        \item алгебраїчні багатосіточні методи~\cite{Stüben-Algebraic-multigrid-AMG-experiences-and-comparisons}, (Algebraic multigrid methods, AMG);
    \end{itemize}
    оптимальні для великих, особливо розріджених матриць, що виникають у дискретизації диференційних систем у часткових похідних.
\end{itemize}
При моделюванні систем у часткових похідних з дрібною сіткою, матриця буде дуже великою і розрідженою -- ми будемо використовувати наближені ітераційні методи.

\textit{Невизначеність системи рівнянь}. Оскільки система невизначена (рівнянь менше ніж змінних), рішень нескінчено багато.
Існують стандартні засоби розв'язку:
\begin{itemize}
    \item Метод найменших квадратів (LSQ) – через QR-розкладання або SVD.
    \item Псевдообернена матриця Мура-Пенроуза~\cite{Penrose-A-generalized-inverse-for-matrices}, реалізація у бібліотеці SciPy: \href{https://docs.scipy.org/doc/scipy/reference/generated/scipy.linalg.pinv.html}{scipy.linalg.pinv()}
    або реалізація у бібліотеці NumPy: \href{https://numpy.org/doc/2.2/reference/generated/numpy.linalg.pinv.html}{numpy.linalg.pinv()}.
\end{itemize}

\textit{Архітектура програмного комплексу}. Оскільки планується віддалений доступ, розрахунки на сервері, робота, яка розрахована на багато користувачів,
то буде використана клієнт-серверна архітектура (сервер на FastAPI/Django, фронт енд на React/Vue, розрахунки на сервері з
використанням NumPy / SciPy або CUDA / OpenCL для прискорення).

\textit{Введення графічної інформації}. Для введення графічної інформації буде використано:
\begin{itemize}
    \item вебінтерфейс з canvas або SVG (зручно для роботи в браузері, можна обробляти форму тіла мишкою, задавати граничні умови кліками);
    \item завантаження файлу з координатами точок і початкових / граничних умов у текстовому форматі.
\end{itemize}
\section{Задача поширення тепла у необмеженій просторово-часовій області} \label{sect4_2}

\textbf{Висновки}. Проведено моделювання динаміки нелінійної системи в необмежених просторово-часових областях.
Встановлена абсолютна величина максимальної похибки моделювання.

\section{Задача поширення тепла в обмеженій просторово-часовій області в умовах
структурної неповноти інформації} \label{sect4_3}

\textbf{Висновки}. Розглянута нелінійна задача поширення тепла в обмеженій просторово-часовій області в умовах
структурної неповноти інформації. Проведено математичне моделювання розв’язків та виконана оцінка похибки у
порівнянні із задачею при повністю заданих початково-крайових умовах.

\section*{Висновки до четвертого розділу}
\addcontentsline{toc}{section}{Висновки до четвертого розділу}

Проведено моделювання динаміки нелінійної системи в необмежених просторово-часових областях. Встановлена абсолютна
величина максимальної похибки моделювання.

Розглянута нелінійна задача поширення тепла в обмеженій просторово-часовій області в умовах структурної
неповноти інформації. Проведено математичне моделювання розв’язків та виконана оцінка похибки у порівнянні із
задачею при повністю заданих початково-крайових умовах.
