\chapter{Математичне моделювання прямих задач динаміки нелінійних систем} \label{ch:chapt2}

В розділі ми розглянули питання побудови середньо квадратичних наближень (якщо вони існують) до розв'язків нелінійних
в малому прямих задач заданих сумою лінійного диференціального оператора та нелінійного збурення в умовах неповноти
інформації.
Нелінійне збурення представлене поліномом зі сталими коефіцієнтами від шуканої функції та її часткових
похідних по координатах.

Наближення до розв'язків прямої нелінійної задачі ми побудували на базі методики розв'язання квазілінійних задач
(метод збурень) описаної у першому розділі даної дисертаційної роботи.

Середньо квадратичні наближення (якщо вони існують) до розв'язків допоміжних прямих лінійних задач ми
побудували на основі методики середньоквадратичного обернення систем функціональних та інтегральних співвідношень
також описаної у першому розділі роботи.

\section{Задача моделювання динаміки нелінійної в малому системи з розподіленими параметрами у необмежених
просторово-часових областях} \label{sec:sect2_1}

Розглянемо систему, яка описується диференціальним рівнянням вигляду
\begin{equation}
    \label{eq:small_parameter_common_system}
L(\partial_x^2, \partial_t)y(s) + \varepsilon P(y(s), \partial_x y(s), \partial_x^2 y(s)) = u(s), s\in S_\infty^\infty
\end{equation}

де $ s=(x, t), x = (x_1,\dots,x_n),
\partial_x=\left(\frac{\partial}{\partial x_1},\dots, \frac{\partial}{\partial x_n}\right),
\partial^2_x=\left(\frac{\partial^2}{\partial x^2_1},\dots, \frac{\partial^2}{\partial x^2_n}\right),\\
n \in \mathbb{N}, \enspace
L(\partial_x^2, \partial_t)
$ -- лінійний диференціальний оператор з постійними коєфіціентеми при похідних
$
L(\partial_x^2, \partial_t)=\sum\limits_{ \substack{i,j=1\\ i\neq j}}\left(l_{1i} \frac{\partial^2}{\partial x^2_i} +
l_{2i}\frac{\partial^2}{\partial x_{1i}
\partial x_{1j}} + l_{3i} \frac{\partial}{\partial x_i} \right) + l_4 \frac{\partial}{\partial t}
$, \enspace
$l_{1i}$, $l_{2i}$, $l_{3i}$, $l_{4}\in\mathbb{R}$,
\enspace $P(y(s), \partial_x y(s), \partial_x^2 y(s))$ -- многочлен своїх аргументів з постійними коефіцієнтами при
похідних $\varepsilon$ -- малий параметр, $\varepsilon\in[0;\enspace0.1]$, $u(s)\in\mathbb{C}$ -- задана функція,
визначена в $S_\infty^\infty$, \enspace$S_\infty^\infty=S_\infty \times T_\infty$, \enspace
$S_\infty = \mathbb{R}^n$, \enspace$T_\infty=
(-\infty;T]$, $T\in\mathbb{R_+}$.

\emph{Наближеним з точністю $\eta$ розв'язком} (якщо він існує) системи \eqref{eq:small_parameter_common_system}, стан
якої визначається у необмеженій просторово-часовій області, ми будемо називати таку функцію $y(s)$,
$s\in S_{\infty}^{\infty}$ що

\begin{equation}
    \label{eq:small_parameter_common_system_solution}
    \int\limits_{\infty}\parallel L(\partial_x^2, \partial_t)y(s) + \varepsilon P(y(s), \partial_x y(s), \partial_x^2 y(s))
    - u (s) \parallel^{2}ds \leq \eta^2.
\end{equation}

Знайдемо $y(s)$. Нехай

\[
y(s)=y^{(0)}(s) + \varepsilon y^{(1)}(s) + \varepsilon^2 y^{(2)}(s) + \dots
\]

де $y^{(0)}(s), y^{(1)}(s), y^{(2)}(s),\dots\in\mathbb{C}^{n}$ -- довільні функції, визначені в $S_{\infty}^{\infty}$.
Підставимо $y(s)$ у вигляді ряду в початкове рівняння~\eqref{eq:small_parameter_common_system}. Маємо

\begin{gather*}
    L(\partial_x^2, \partial_t)y(s) + \varepsilon P(y(s), \partial_x y(s), \partial_x^2 y(s)) = u(s) \Leftrightarrow\\
    L(\partial_x^2, \partial_t)y^{(0)}(s) - u(s) +
    \varepsilon \left\{ P_{1}
                    \left[
                       y^{(0)}(s), \partial_x y^{(0)}(s), \partial_x^2 y^{(0)}(s)
                    \right] +
                    L(\partial_x^2, \partial_t)y^{(1)}(s)
                \right\} +\\
    + \varepsilon^{2} \left\{ P_{2}
\left[
y^{(0)}(s), \partial_x y^{(0)}(s), \partial_x^2 y^{(0)}(s),
y^{(1)}(s), \partial_x y^{(1)}(s), \partial_x^2 y^{(1)}(s)
\right] +
L(\partial_x^2, \partial_t)y^{(2)}(s)
\right\} +\\
    \dots\\
    + \varepsilon^{q} \left\{ P_{q}
\left[
\underbrace{
    ,\dots,
    y^{(q-1)}(s), \partial_x y^{(q-1)}(s), \partial_x^2 y^{(q-1)}(s)
}_{3q}
\right]
+ L(\partial_x^2, \partial_t)y^{(q)}(s)
\right\} + \dots = 0\\
\end{gather*}

де $P_{1}[\cdot]$, $P_{2}[\cdot]$, $P_{q}[\cdot]$ -- деякі многочлени, $q=1,2,\dots$.
Оскільки послідовність степенів $\varepsilon$ лінійно незалежна, то коефіцієнт при кожній степені обертається в нуль
незалежно.
Отже, можемо вимагати, щоб

\begin{gather*}
    L(\partial_x^2, \partial_t)y^{(0)}(s) - u(s) = 0,\\
    L(\partial_x^2, \partial_t)y^{(1)}(s) = -P_{1}
    \left[
    y^{(0)}(s), \partial_x y^{(0)}(s), \partial_x^2 y^{(0)}(s)
    \right], \\
    L(\partial_x^2, \partial_t)y^{(2)}(s) = -P_{2}
    \left[
    y^{(0)}(s), \partial_x y^{(0)}(s), \partial_x^2 y^{(0)}(s),
    y^{(1)}(s), \partial_x y^{(1)}(s), \partial_x^2 y^{(1)}(s)
    \right],\\
    \dots\\
    L(\partial_x^2, \partial_t)y^{(q)}(s) = -P_{q}
    \left[
        \underbrace{
            ,\dots,
            y^{(q-1)}(s), \partial_x y^{(q-1)}(s), \partial_x^2 y^{(q-1)}(s)
        }_{3q}
        \right].
\end{gather*}

Позначимо $y^{q}(s)=y_{\infty}^{q}(s)$.
Розглянемо рівняння $L(\partial_x^2, \partial_t)y^{(0)}(s) - u(s) = 0$.
Відомо, якщо $y(s)\in\mathbb{C}^{n}$, то функію $y(s)$ можна представити у вигляді

\[
    y(s) = \int\limits_{S_{\infty}^{\infty}}^{}y(s')\delta(s-s')ds',
\]

де $ds'=dx_{1}'\dots dx_{n}'dt'$, $\delta(s-s')= \prod\limits_{k=1}^{n}\delta(x_{k} - x_{k}')\delta(t-t')$,
$\delta(\cdot)$ -- дельта функція Дірака, яку ми запишемо у вигляді
\begin{gather*}
    \delta(x_{k} - x'_{k})=\frac{1}{2\pi}\int\limits_{-\infty}^{\infty}\exp(i\lambda_{k}(x_{k}-x'_{k}))d\lambda_{k},\\
    \delta(t - t')=\frac{1}{2\pi}\int\limits_{-\infty}^{\infty}\exp(i\mu(t-t'))d\mu.\\
\end{gather*}
Тоді
\[
    y^{(0)}_{\infty}(0)=\int\limits_{S_{\infty}^{\infty}}G(s-s')ds'.
\]

де $G(s-s')$ -- функція Гріна для необмеженій просторово-часовій області, для рівняння
$L(\partial_x^2, \partial_t)y^{(0)}(s) - u(s) = 0$.
Відомо, що

\begin{gather*}
    G(x-x', t-t') = \frac{1}{(2\pi)^{n+1}}\int\limits_{S_\infty^\infty}Q(-\lambda^2, i\mu)\prod_{k=1}^n\exp
    (i\lambda_k(x_k - x'_k))\exp(i\mu(t-t'))d\lambda d\mu,\\
    \lambda=(\lambda_{1},\dots,\lambda_{n}),
\end{gather*}
де $i$ -- уявна одиниця, $\lambda_{1},\dots,\lambda_{n},\mu\in\mathbb{R},s=(x,t),x=(x_{1},\dots,x_{n}), n\in\mathbb{N}$.
Розглянемо рівняння $L(\partial_x^2, \partial_t)y^{(1)}(s)=-P_{1}
\left[
y^{(0)}(s), \partial_x y^{(0)}(s), \partial_x^2 y^{(0)}(s)
\right]$.
Оскільки $y^{(0)}(s)=y^{(0)}_{\infty}(s)$ вже відомо, то
\begin{gather*}
    L(\partial_x^2, \partial_t)y^{(1)}(s)=-P_{1}
    \left[
    y_{\infty}^{(0)}(s), \partial_x y_{\infty}^{(0)}(s), \partial_x^2 y_{\infty}^{(0)}(s)
    \right].
\end{gather*}
Отже,
\[
    y_{\infty}^{(1)}(s) = -\int\limits_{S_\infty^\infty}G(s-s')P_{1}
    \left[
    y_{\infty}^{(0)}(s'), \partial_x y_{\infty}^{(0)}(s'), \partial_x^2 y_{\infty}^{(0)}(s')
    \right]ds'.
\]
Аналогічно
\begin{gather*}
    y_{\infty}^{(2)}(s) = -\int\limits_{S_\infty^\infty}G(s-s')\times\\
    \times P_{2}
    \left[
    y_{\infty}^{(0)}(s'), \partial_x y_{\infty}^{(0)}(s'), \partial_x^2 y_{\infty}^{(0)}(s'),
    y_{\infty}^{(1)}(s'), \partial_x y_{\infty}^{(1)}(s'), \partial_x^2 y_{\infty}^{(0)}(s')
    \right]ds'\\
    \dots\\
    y_{\infty}^{(q)}(s)=-\int\limits_{S_\infty^\infty}G(s-s')\times\\
    \times P_{2}
    \left[
        \underbrace{
            ,\dots,
            y^{(q-1)}(s), \partial_x y^{(q-1)}(s), \partial_x^2 y^{(q-1)}(s)
        }_{3q}
    \right]ds'\\
    \dots
\end{gather*}
Остаточно
\begin{gather*}
    y(s)=y_{\infty}^{(0)}(s) + \varepsilon y_{\infty}^{(1)}(s) + \varepsilon^2 y_{\infty}^{(2)}(s) + \dots
\end{gather*}

\textbf{Висновки}.
Розглянуто нелінійну систему, яка функціонує у нескінчених просторово-часових областях.
Отримано наближення нелінійної системи через нескінчену кількість лінійних систем.
Для кожної з лінійних систем побудовано функції Гріна.
Проведено моделювання розв'язків лінійних систем та отримано розв'язок початкової системи у вигляді ряду.

\section{Задача моделювання динаміки нелінійної в малому системи з розподіленими параметрами у обмеженій
часовій області} \label{sec:sect2_2}

Розглянемо систему, яка описується диференціальним рівнянням з початковою умовою, заданою на всій області
$S_{\infty}$

\begin{equation}
    \label{eq:small_parameter_common_system_with_s_area}
    \left\{
    \begin{alignedat}{2}
    L(\partial_x^2, \partial_t)y(s) + \varepsilon P(y(s), \partial_x y(s), \partial_x^2 y(s)) = u(s), s\in S_\infty^T, \\
    L^{0}(\partial_{t})y(x,0) = Y^{0}(x),x\in\bigcup\limits_{r=1}^{R_{0}} S_{0r},t=0,
    \end{alignedat}
    \right.
\end{equation}
де $s=(x,t)$, $x=(x_{1},\dots,x_{n})$, $S_{\infty}^{T}=S_{\infty}\times T_{0}$, $\partial_{x}=\left(
\frac{\partial}{\partial x_{1}}, \dots, \frac{\partial}{\partial x_{n}} \right)$,
$\partial^{2}_{x}=\left(\frac{\partial^{2}}{\partial x^{2}_{1}}, \dots, \frac{\partial^{2}}{\partial x^{2}_{n}} \right)$,
$L(\partial_x^2, \partial_t)$, $L^{0}(\partial_{t})$ -- лінійні диференціальні оператори з постійними коефіцієнтами при похідних,
$
L(\partial_x^2, \partial_t)=\sum\limits_{ \substack{i,j=1\\ i\neq j}}\left(l_{1i} \frac{\partial^2}{\partial x^2_i} +
l_{2i}\frac{\partial^2}{\partial x_{1i}
    \partial x_{1j}} + l_{3i} \frac{\partial}{\partial x_i} \right) + l_4 \frac{\partial}{\partial t}
$, $l_{1i}$, $l_{2i}$, $l_{3i}$, $l_{4}\in\mathbb{R}$, $L^{0}(\partial_{t})=l_{1}^{0}\frac{\partial}{\partial t}$,
$l_{1}^{0}\in\mathbb{R}$, $P(y(s), \partial_x y(s), \partial_x^2 y(s))$ -- многочлен  своїх аргументів з постійними
коефіцієнтами при похідних,  $\varepsilon$ – малий параметр, $\varepsilon\in [0; 0.01]$,
$Y_{0}(x)\in\mathbb{C}^n$ -- задана функція, визначена в $\bigcup\limits_{r=1}^{R_{0}}$;
$S_{0r}\subseteq S_{0}$; $S_{0}$ -- зв'язна замкнена множина; $S_{0r}\cap S_{ok} = \varnothing$; $r\neq k$;
$r,k = 1,\dots,R_{0}$; $R_{0}\in\mathbb{N}$; $u(s)\in\mathbb{C}^n$ -- задана функція, визначена в
$S_{\infty}^{T}$, $S_{\infty} = \mathbb{R}$, $k,n\in\mathbb{N}$, $T_{0}=[0;T]$, $T\in\mathbb{R}_{+}$.
Використовуючи заміну
$$y(s)=y^{(0)}(s)+\varepsilon y^{(1)}(s) + \varepsilon^{2}y^{(2)}(s)+\dots$$
отримаємо
\begin{gather*}
    \left\{
    \begin{alignedat}{2}
        L(\partial_x^2, \partial_t)y(s) - u(s)
        + \varepsilon \left\{ P_{1}\left[y^{(0)}(s), \partial_x y^{(0)}(s), \partial_x^2 y^{(0)}(s)\right]
        + L(\partial_x^2, \partial_t)y^{(1)}(s) \right\} + \\
        \dots\\
        + \varepsilon^{q} \left\{ P_{q}
        \left[
            \underbrace{
                ,\dots,
                y^{(q-1)}(s), \partial_x y^{(q-1)}(s), \partial_x^2 y^{(q-1)}(s)
            }_{3q}
            \right]
        + L(\partial_x^2, \partial_t)y^{(q)}(s)
        \right\} + \dots = 0,\\
        L^{0}(\partial_t)y^{(0)}(x, 0) + \varepsilon L^{0}(\partial_t)y^{(1)}(x, 0)+\dots+
        \varepsilon^q L^{0}(\partial_t)y^{(1)}(x, 0)+\dots = Y^{0}(x),
    \end{alignedat}
    \right.
\end{gather*}
де $P_{1}[\cdot], P_{2}[\cdot], \dots,P_{q}[\cdot],\dots$ -- деякі многочлени; $q=1,2,\dots$.
Отже,
\begin{gather*}
    \left\{
    \begin{alignedat}{2}
        L(\partial_x^2, \partial_t)y^{(0)}(s) - u(s) = 0, s\in S_\infty^T,\\
        L^{0}(\partial_{t})y^{(0)}(x,0) = Y^{0}(x),x\in\bigcup\limits_{r=1}^{R_{0}} S_{0r},t=0,
    \end{alignedat}
    \right.\\
    \left\{
    \begin{alignedat}{2}
        L(\partial_x^2, \partial_t)y^{(1)}(s) = -P_{1}\left[y^{(0)}(s), \partial_x y^{(0)}(s), \partial_x^2 y^{(0)}(s)\right], s\in S_\infty^T,\\
        L^{0}(\partial_{t})y^{(1)}(x,0) = Y^{0}(x),x\in\bigcup\limits_{r=1}^{R_{0}} S_{0r},t=0,
    \end{alignedat}
    \right.\\
    \dots\\
    \left\{
    \begin{alignedat}{2}
        L(\partial_x^2, \partial_t)y^{(q)}(s) =
        -P_{q}
        \left[
            \underbrace{
                ,\dots,
                y^{(q-1)}(s), \partial_x y^{(q-1)}(s), \partial_x^2 y^{(q-1)}(s)
            }_{3q}
            \right], s\in S_\infty^T,\\
        L^{0}(\partial_{t})y^{(q)}(x,0) = Y^{0}(x),x\in\bigcup\limits_{r=1}^{R_{0}} S_{0r},t=0,
    \end{alignedat}
    \right.
\end{gather*}

Розглянемо систему

\begin{gather*}
    \left\{
    \begin{alignedat}{2}
        L(\partial_x^2, \partial_t)y^{(0)}(s) - u(s) = 0, s\in S_\infty^T,\\
        L^{0}(\partial_{t})y^{(0)}(x,0) = Y^{0}(x),x\in\bigcup\limits_{r=1}^{R_{0}} S_{0r},t=0,
    \end{alignedat}
    \right.
\end{gather*}

Нехай $y^{(0)}(s)=y_{\infty}^{(0)}(s) + y_{0}^{(0)}(s)$,
де $y_{0}^{(0)}(s)$ -- функція, яка виражається через фіктивні динамічні збурення
$u^{(0)}_{0}(s')$,
$s=(x', t')$,
$s'\in S_{\infty}^{\infty 0} = S_{\infty} \times (T_{\infty}\textbackslash T_{0})$,
$y_{0}^{(0)}(s)=\int\limits_{S_{\infty}^{\infty 0}}G(s-s')\bigg|_{s\in S_{\infty}^{T}}u_{0}^{(0)}(s')ds'$,
$y_{\infty}^{(0)}(s)=\int\limits_{S_{\infty}^{\infty}}G(s-s')\bigg|_{s\in S_{\infty}^{T}}u_{0}^{(0)}(s')ds'$,
$ds'=dx'dt'$,
де $G(s-s')$ -- функція Гріна для необмеженій просторово-часовій області, для рівняння
$L(\partial_x^2, \partial_t)y^{(0)}(s) - u(s) = 0$.

Маємо рівняння відносно $y_{0}^{(0)}$:

\begin{gather*}
L^{0}(\partial_{t})y_{\infty}^{(0)}(s) + L^{0}(\partial_{t})y_{\infty}^{(0)}(s) = Y^{0}(x),
x\in\bigcup\limits_{r=1}^{R_{0}} S_{0r},t=0,\Leftrightarrow\\
    \int\limits_{S_{\infty}^{\infty 0}}L^{0}(\partial_{t})G(s-s')u_{0}^{(0)}(s')ds' =
    Y^{0}(x)-\int\limits_{S_{\infty}^{\infty}} L^{0}(\partial_{t})G(s-s')u_{0}^{(0)}(s')ds',\\
x\in\bigcup\limits_{r=1}^{R_{0}} S_{0r},t=0.
\end{gather*}

Дискретизуючи по змінній $s'$, замінимо наступні інтеграли інтегральними сумами:

\begin{gather*}
\int\limits_{S_{\infty}^{\infty 0}}L^{0}(\partial_{t})G(s-s')u_{0}^{(0)}(s')ds' \approx
\sum\limits_{\xi=1}^{\Xi} L^{0}(\partial_{t})G(s-s'_{\xi})u_{0}^{(0)}(s'_{\xi})\Delta s'_{\xi},\\
x\in\bigcup\limits_{r=1}^{R_{0}} S_{0r},t=0.
\end{gather*}

Отже,

\begin{gather*}
    \sum\limits_{\xi=1}^{\Xi} L^{0}(\partial_{t})G(s-s'_{\xi})\bigg|_{s'_{\xi}\in S_{\infty}^{\infty 0}}
    u_{0}^{(0)}(s'_{\xi})\Delta s'_{\xi}
    = Y^{0}(x) - \int\limits_{S_{\infty}^{\infty}} L^{0}(\partial_{t})G(s-s') \bigg|_{s\in S_{\infty}^{T}} u_{0}^{(0)}(s')ds'\\
    x\in\bigcup\limits_{r=1}^{R_{0}} S_{0r},t=0.
\end{gather*}

Дискретизуючи по змінній $s$, замінимо  рівняння  системою з $H_{0}$ лінійних алгебраїчних рівнянь.

\begin{equation}
    \begin{alignedat}{2}
    \label{eq:small_parameter_common_system_with_s_area_summa}
    \sum\limits_{\xi=1}^{\Xi} L^{0}(\partial_{t})G(s_{h}-s'_{\xi})\bigg|_{s'_{\xi}\in S_{\infty}^{\infty 0}}
    u_{0}^{(0)}(s'_{\xi})\Delta s'_{\xi} =\\
    = Y^{0}(x_{h}) - \int\limits_{S_{\infty}^{\infty}} L^{0}(\partial_{t})G(s_{h}-s') \bigg|_{s_{h}\in S_{\infty}^{T}} u_{0}^{(0)}(s')ds',\\
    x_{h}\in\bigcup\limits_{r=1}^{R_{0}} S_{0r},t=0.
    \end{alignedat}
\end{equation}

Запишемо систему ($\ref{eq:small_parameter_common_system_with_s_area_summa}$) у матричній формі

\begin{gather*}
    CU_{0}^{0}=Y_{0}^{(0)},\\
    U_{0}^{(0)} = \left( str \left( u_{0}^{(0)}(s'_{\xi})\bigg|_{s'_{\xi}\in S_{\infty}^{\infty 0}} \right) \xi,\dots,{\Xi} \right)^{T},\\
    Y_{0}^{(0)}=col\left(Y^{0}(x_{h}) - \int\limits_{S_{\infty}^{\infty}} L_{0}(\partial_{t})G(s_{h}-s')\bigg|_{s_{h}\in S_{\infty}
    ^{T}}U(s')ds', x_{h}\in\bigcup\limits_{r=1}^{R_{0}} S_{0r},t=0 \right)\\ h=1,\dots,H_{0},\\
    C=col\left( str\left(L^{0}(\partial_{t})G(s_{h}-s'_{\xi})\bigg|_{\substack{s_{h}\in\bigcup
    \limits_{r=1}^{R_{0}} S_{0r}\\
    s'_{\xi}\in S_{\infty}^{\infty 0}}} \Delta s'_{\xi} \right)\xi,\dots,{\Xi} \right)\\ h=1,\dots,H_{0}.
\end{gather*}

Отримаємо задачу псевдообернення системи лінійних алгебраїчних рівнянь.
Позначимо $P_{1}=CC^{T}$, $P_{2}=C^{T}C$. Запишемо її розв'язок.

\begin{gather*}
    U_{0}^{(0)}\in \Omega_{U_{0}^{(0)}} = \left\{U_{0}^{(0)} : U_{0}^{(0)} = C^{T}P_{1}^{+}Y_{0}^{(0)}
    + I_{\Xi\times\Xi}\nu -C^{T}P_{1}^{+}C\nu\right\},
\end{gather*}

або

\begin{gather*}
    U_{0}^{(0)}\in \Omega_{U_{0}^{(0)}} = \left\{U_{0}^{(0)} : U_{0}^{(0)} = P_{2}^{+}C^{T}Y_{0}^{(0)}
    + I_{\Xi\times\Xi}\nu -P_{2}^{+}C^{T}C\nu\right\},
\end{gather*}

де $I_{\Xi\times\Xi}$ -- одинична матриця розміру $\Xi\times\Xi$, $\nu$ -- довільний вектор розміру $\Xi$,
$P_{1}^{+}$, $P_{2}^{+}$  -- матриці псевдообернені до $P_{1}$,$P_{2}$. Похибка моделювання $\left[ \eta^{(0)} \right]^{2}$ дорівнює

\begin{gather*}
    \left[ \eta^{(0)} \right]^{2} = \left[ Y_{0}^{(0)} \right]^{T}Z\left( C^{T} \right)Y_{0}^{(0)},
    Z\left( C^{T} \right)=I_{H_{0}\times H_{0}} - C\left( C^{T} C\right)^{+}C^{T}.
\end{gather*}

Таким чином ми знайшли $y^{(0)}(s)$. Підставляючи значення $y^{(0)}(s)$ в систему (\ref{eq:small_parameter_common_system_with_s_area_summa_iteratoin_1})

\begin{equation}
    \left\{
    \label{eq:small_parameter_common_system_with_s_area_summa_iteratoin_1}
    \begin{alignedat}{2}
        L(\partial_x^2, \partial_t)y^{(1)}(s) = -P_{1}\left[ y^{0}(0), \partial_{x}
        y^{(0)}(s), \partial_{x}^{2}y^{(0)}(s) \right], s\in S_\infty^T,\\
        L^{0}(\partial_{t})y^{(1)}(x,0) = Y^{0}(x),x\in\bigcup\limits_{r=1}^{R_{0}} S_{0r},t=0,
    \end{alignedat}
    \right.
\end{equation}

визначимо $y^{(0)}(s)$.
В загальному випадку для системи

\begin{gather*}
    \left\{
    \begin{alignedat}{2}
        L(\partial_x^2, \partial_t)y^{(q)}(s) = -P_{q}
        \left[
            \underbrace{
                ,\dots,
                y^{(q-1)}(s), \partial_x y^{(q-1)}(s), \partial_x^2 y^{(q-1)}(s)
            }_{3q}
            \right], s\in S_\infty^T,\\
        L^{0}(\partial_{t})y^{(q)}(x,0) = Y^{0}(x),x\in\bigcup\limits_{r=1}^{R_{0}} S_{0r},t=0,
    \end{alignedat}
    \right.
\end{gather*}

позначимо $y^{(q)}(s)=y_{\infty}^{(q)}(s) + y_{0}^{(q)}(s)$,
де $y_{0}^{(q)}(s)$ -- функція, яка виражається через фіктивні динамічні збурення
$u^{(q)}_{0}(s')$,
$s'=(x', t')$,
$s'\in S_{\infty}^{\infty 0} = S_{\infty} \times (T_{\infty}\textbackslash T_{0})$,
$y_{q}^{(0)}(s)=\int\limits_{S_{\infty}^{\infty 0}}G(s-s')\bigg|_{s\in S_{\infty}^{T}}u_{0}^{(q)}(s')ds'$,
$y_{\infty}^{(q)}(s)=\int\limits_{S_{\infty}^{\infty}}G(s-s')\bigg|_{s\in S_{\infty}^{T}}P_{q}\left[ \cdot \right](s')ds'$,
$ds'=dx'dt'$.

Отже,

\begin{gather*}
   \int\limits_{S_{\infty}^{\infty 0}} L^{0}(\partial_{t})G(s-s')u_{0}^{(q)}(s')ds' =
   \int\limits_{S_{\infty}^{\infty}}
   L^{0}(\partial_{t})G(s-s')P_{q}\left[ \cdot \right]ds',\\
   x\in\bigcup\limits_{r=1}^{R_{0}} S_{0r},t=0.
\end{gather*}

Аналогічно, дискретизуючи по змінній $s'$, замінимо інтеграли інтегральними сумами;
дискретизуючи по змінній $s$, замінимо  рівняння  системою з $H_{0}$ лінійних алгебраїчних рівнянь.
Запишемо отриману систему у матричній формі

\begin{gather*}
    CU_{0}^{q}=Y_{0}^{(q)},\\
    C=col\left( str\left(L^{0}(\partial_{t})G(s_{h}-s'_{\xi})\bigg|_{\substack{s_{h}\in\bigcup
    \limits_{r=1}^{R_{0}} S_{0r}\\
    s'_{\xi}\in S_{\infty}^{\infty 0}}} \Delta s'_{\xi} \right)\xi,\dots,{\Xi} \right)\\ h=1,\dots,H_{0},\\
    U_{0}^{(q)} = \left( str \left( u_{0}^{(q)}(s'_{\xi})\bigg|_{s'_{\xi}\in S_{\infty}^{\infty 0}} \right) \xi,\dots,{\Xi} \right)^{T},\\
    Y_{0}^{(q)}=col\left(\int\limits_{S_{\infty}^{\infty}} L_{0}(\partial_{t})G(s_{h}-s')\bigg|_{s_{h}\in S_{\infty}
    ^{T}}P_{q}\left[ \cdot \right]ds', x_{h}\in\bigcup\limits_{r=1}^{R_{0}} S_{0r},t=0 \right)\\ h=1,\dots,H_{0},\\
\end{gather*}

Позначимо $P_{1}=CC^{T}$, $P_{2}=C^{T}C$.
Запишемо її розв'язок.

\begin{gather*}
    U_{0}^{(q)}\in \Omega_{U_{0}^{(q)}} = \left\{U_{0}^{(q)} : U_{0}^{(q)} = C^{T}P_{1}^{+}Y_{0}^{(q)}
    + I_{\Xi\times\Xi}\nu -C^{T}P_{1}^{+}C\nu\right\},
\end{gather*}

або

\begin{gather*}
    U_{0}^{(q)}\in \Omega_{U_{0}^{(q)}} = \left\{U_{0}^{(q)} : U_{0}^{(q)} = P_{2}^{+}C^{T}Y_{0}^{(q)}
    + I_{\Xi\times\Xi}\nu -P_{2}^{+}C^{T}C\nu\right\},
\end{gather*}

де $I_{\Xi\times\Xi}$ -- одинична матриця розміру $\Xi\times\Xi$, $\nu$ -- довільний вектор розміру $\Xi$,
$P_{1}^{+}$, $P_{2}^{+}$  -- матриці псевдообернені до $P_{1}$,$P_{2}$.
Похибка моделювання $\left[ \eta^{(q)} \right]^{2}$ дорівнює

\begin{gather*}
    \left[ \eta^{(q)} \right]^{2} = \left[ Y_{0}^{(q)} \right]^{T}Z\left( C^{T} \right)Y_{0}^{(q)},
    Z\left( C^{T} \right)=I_{H_{0}\times H_{0}} - C\left( C^{T} C\right)^{+}C^{T}.
\end{gather*}

Загальна похибка моделювання дорівнює
\begin{gather*}
    \eta^{2} = \left[ \eta^{(0)} \right]^{2} + \left[ \eta^{(1)} \right]^{2} + \dots + \left[ \eta^{(q)} \right]^{2} + \dots
\end{gather*}

\textbf{Висновки}.
Розглянуто нелінійну систему, яка функціонує у скінченій часовій області та нескінченій
просторовій області.
Отримано наближення нелінійної системи через нескінчену кількість лінійних систем.
Для кожної з лінійних систем побудовано функції Гріна.
Проведено моделювання розв'язків лінійних систем та отримано розв'язок початкової системи у вигляді ряду.
Знайдено похибку моделювання для нелінійних систем, що функціонують у скінченій часовій області.

\section{Задача моделювання динаміки нелінійної в малому системи з розподіленими параметрами у обмеженій
просторовій області} \label{sec:sect2_3}

Нехай система описується диференційним рівнянням з граничною умовою

\begin{equation}
    \label{eq:small_parameter_common_system_with_conditions}
    \left\{
    \begin{alignedat}{2}
        L(\partial_x^2, \partial_t)y(s) + \varepsilon P(y(s), \partial_x y(s), \partial_x^2 y(s)) = u(s), s\in S_{0}^T, \\
        L^{\Gamma}(\partial^{2}_{x})y(s) = Y^{\Gamma}(s),s\in\bigcup\limits_{\rho=1}^{R_{\Gamma}} \Gamma_{0\rho}\times T_{0},
    \end{alignedat}
    \right.
\end{equation}
де $s=(x,t)$, $x=(x_{1},\dots,x_{n})$, $S_{0}^{T}=S_{0}\times T_{0}$, $\partial_{x}=\left(
\frac{\partial}{\partial x_{1}}, \dots, \frac{\partial}{\partial x_{n}} \right)$,
$\partial^{2}_{x}=\left(\frac{\partial^{2}}{\partial x^{2}_{1}}, \dots, \frac{\partial^{2}}{\partial x^{2}_{n}} \right)$,
$L(\partial_x^2, \partial_t)$, $L^{\Gamma}(\partial_{x})$ -- лінійні диференціальні оператори з постійними коефіцієнтами при похідних,
$
L(\partial_x^2, \partial_t)=\sum\limits_{ \substack{i,j=1\\ i\neq j}}\left(l_{1i} \frac{\partial^2}{\partial x^2_i} +
l_{2i}\frac{\partial^2}{\partial x_{1i}
    \partial x_{1j}} + l_{3i} \frac{\partial}{\partial x_i} \right) + l_4 \frac{\partial}{\partial t}
$,
$
L^{\Gamma}(\partial_x^2)=\sum\limits_{ \substack{i,j=1\\ i\neq j}}\left(l^{\Gamma}_{1i} \frac{\partial^2}{\partial x^2_i} +
l^{\Gamma}_{2i}\frac{\partial^2}{\partial x_{1i}
    \partial x_{1j}} + l^{\Gamma}_{3i} \frac{\partial}{\partial x_i} \right)
$,
$l_{1i}$, $l_{2i}$, $l_{3i}$, $l_{4}\in\mathbb{R}$,
$l^{\Gamma}_{1i}$, $l^{\Gamma}_{2i}$, $l^{\Gamma}_{3i}$,
$P(y(s), \partial_x y(s), \partial_x^2 y(s))$ -- многочлен  своїх аргументів з постійними
коефіцієнтами при похідних,  $\varepsilon$ – малий параметр, $\varepsilon\in [0; 0.01]$,
$Y^{\Gamma}(x)\in\mathbb{C}^n$ -- задана функція, визначена в
$\bigcup\limits_{\rho=1}^{R_{\Gamma}}\Gamma_{0\rho}\times T_{0}$;
$\Gamma_{0\rho}\subseteq \Gamma_{0}$; $\Gamma_{0\rho}\cap\Gamma_{0\sigma} = \varnothing$;
$\rho\neq\sigma$; $\rho,\sigma = 1,\dots,R_{\Gamma}$, $R_{\Gamma}\in\mathbb{N}$; $u(s)\in\mathbb{C}^n$ --
задана функція, визначена в $S_{0}^{T}$, $S\subset S_{\infty}=\mathbb{R}^{n}$,
$S_{0}$ -- зв'язна замкнена множина; $\Gamma$ -- границя області $S_{0}$, $k,n\in\mathbb{N}$, $T_{0}=[0;T]$,
$T\in\mathbb{R}_{+}$.

Використовуючи розглянуту вище методику, запишемо систему рівнянь

\begin{gather*}
    \left\{
    \begin{alignedat}{2}
        L(\partial_x^2, \partial_t)y^{(0)}(s) - u(s) = 0, s\in S_{0}^T, \\
        L^{\Gamma}(\partial_{x})y^{(0)}(s) = Y^{\Gamma}(s),s\in\bigcup\limits_{\rho=1}^{R_{\Gamma}} \Gamma_{0\rho}\times T_{0},
    \end{alignedat}
    \right.\\
    \left\{
    \begin{alignedat}{2}
        L(\partial_x^2, \partial_t)y^{(1)}(s) = - P_{1}\left[
        y^{(0)}(s), \partial_x y^{(0)}(s), \partial_x^2 y^{(0)}(s)
        \right], s\in S_{0}^T, \\
        L^{\Gamma}(\partial_{x})y^{(1)}(s) = 0,s\in\bigcup\limits_{\rho=1}^{R_{\Gamma}} \Gamma_{0\rho}\times T_{0},
    \end{alignedat}
    \right.\\
    \dots\\
    \left\{
    \begin{alignedat}{2}
        L(\partial_x^2, \partial_t)y^{(q)}(s) = - P_{q}\left[
            \underbrace{
                ,\dots,
                y^{(q-1)}(s), \partial_x y^{(q-1)}(s), \partial_x^2 y^{(q-1)}(s)
            }_{3q}
        \right], s\in S_{0}^T, \\
        L^{\Gamma}(\partial_{x})y^{(q)}(s) = 0,s\in\bigcup\limits_{\rho=1}^{R_{\Gamma}} \Gamma_{0\rho}\times T_{0},
    \end{alignedat}
    \right.
\end{gather*}

Розглянемо систему

\begin{gather*}
    \left\{
    \begin{alignedat}{2}
        L(\partial_x^2, \partial_t)y^{(0)}(s) - u(s) = 0, s\in S_{0}^T, \\
        L^{\Gamma}(\partial_{x})y^{(0)}(s) = Y^{\Gamma}(s),s\in\bigcup\limits_{\rho=1}^{R_{\Gamma}} \Gamma_{0\rho}\times T_{0},
    \end{alignedat}
    \right.\\
\end{gather*}

Нехай $y^{(0)}(s)=y^{(0)}_{\infty}(s) + y^{(0)}_{\Gamma}(s)$,
де $y^{(0)}_{\Gamma}(s)$ -- функція, яку можемо виразити через фіктивні динамічні збурення
$u^{(0)}_{\Gamma}(s')$, $s'=(x', t')$, $s'\in S^{T}_{\infty 0}=(S_{\infty} \textbackslash S_{0})\times T_{0}$,
$y^{(0)}_{\Gamma}(s) = \int\limits_{S^{T}_{\infty 0}}G(s-s')\bigg|_{s\in S^{T}_{0}}u^{(0)}_{\Gamma}(s')ds'$,
$y^{(0)}_{\infty}(s) = \int\limits_{S^{\infty}_{\infty}}G(s-s')\bigg|_{s\in S^{T}_{0}}u^{(0)}_{\Gamma}(s')ds'$,
$ds'=dx'dt'$, де $G(s-s')$ -- функція Гріна для необмеженій просторово-часовій області, для рівняння
$L(\partial^{2}_{x}, \partial_{t})y^{(0)}(s) - u(s) = 0$.

Отримаємо інтегральне рівняння відносно $y^{(0)}_{\Gamma}(s)$.

\begin{gather*}
L^{\Gamma}(\partial_{x})y^{(0)}_{\infty}(s) + L^{\Gamma}(\partial_{x})y^{(0)}_{\Gamma}(s) = Y^{\Gamma}(s),
s\in\bigcup\limits_{\rho=1}^{R_{\Gamma}} \Gamma_{0\rho}\times T_{0} \Leftrightarrow\\
    \int\limits_{S^{\infty}_{\infty}}L^{\Gamma}(\partial_{x})G(s-s')u(s')ds' +\\
    + \int\limits_{S^{T}_{\infty 0}}G(s-s')L^{\Gamma}(\partial_{x})u^{(0)}_{\Gamma}(s')ds' = Y^{\Gamma}(s),
s\in\bigcup\limits_{\rho=1}^{R_{\Gamma}} \Gamma_{0\rho}\times T_{0}.
\end{gather*}

Дискретизуючи по змінній $s'$ та $s$, замінимо інтеграли інтегральними сумами, а  рівняння -- системою з $W_{\Gamma}$
лінійних алгебраїчних рівнянь:

\begin{equation}
    \label{eq:small_parameter_common_system_with_conditions_in_descrete_dots}
    \begin{alignedat}{2}
    \sum_{\varphi=1}^{\Phi} L^{\Gamma}(\partial_{x})G(s_{w} - s'_{\varphi})
    \bigg|_{\substack{
    s_{w}\in
    \bigcup\limits_{\rho=1}^{R_{\Gamma}} \Gamma_{0\rho}\times T_{0}\\
    s'_{\varphi}\in S^{T}_{\infty 0}
    }
    }
    u^{(0)}_{\Gamma}(s'_{\varphi})\Delta s'_{\varphi} =\\
    = Y^{\Gamma}(s_{w})-\int\limits_{S^{\infty}_{\infty}} L^{\Gamma}(\partial_{x})G(s_{w}-s')
    \bigg|_{s_{w}\in\bigcup\limits_{\rho=1}^{R_{\Gamma}} \Gamma_{0\rho}\times T_{0}}u(s')ds',\\
    s_{w}\in\bigcup\limits_{\rho=1}^{R_{\Gamma}} \Gamma_{0\rho}\times T_{0}, w=1,2,\dots,W_{\Gamma}.
    \end{alignedat}
\end{equation}

Запишемо систему (\ref{eq:small_parameter_common_system_with_conditions_in_descrete_dots}) у матричній формі

\begin{gather*}
    CU^{(0)}_{\Gamma} = Y^{(0)}_{\Gamma},\\
    U^{(0)}_{\Gamma}=\left(str\left( u^{(0)}_{\Gamma}
    (s'_{\varphi})\bigg|_{s'_{\varphi}\in S^{T}_{\infty 0}} \right) \varphi=1,\dots,\Phi \right)^{T}\\
    Y^{(0)}_{\Gamma}=col\left(Y^{\Gamma}(s_{w})
    -\int\limits_{S^{\infty}_{\infty}}L^{\Gamma}(\partial_{x})G(s_{w}-s')\bigg|_{s\in\Gamma^{T}_{0}}u(s')ds',
    s_{w}\in\bigcup\limits_{\rho=1}^{R_{\Gamma}} \Gamma_{0\rho}\times T_{0}
    \right)\\
    w=1,\dots,W_{\Gamma}\\
    C=col\left( str\left(L^{\Gamma}(\partial_{x}) G(s_{w}-s'_{\varphi})
    \bigg|_{\substack{
        s_{w}\in
        \bigcup\limits_{\rho=1}^{R_{\Gamma}} \Gamma_{0\rho}\times T_{0}\\
        s'_{\varphi}\in S^{T}_{\infty 0}
    }}
    \right)\varphi=1,\dots,\Phi \right)\\
    w=1,\dots,W_{\Gamma}
\end{gather*}

Отримаємо задачу псевдообернення системи лінійних алгебраїчних рівнянь.
Позначимо $P_{1}=CC^{T}$, $P_{2}C^{T}C$.
Запишемо її розв'язок.

\begin{gather*}
U^{(0)}_{\Gamma}\in\Omega_{U^{(0)}_{\Gamma}}=
    \left\{ U^{(0)}_{\Gamma}:U^{(0)}_{\Gamma}=C^{T}P^{+}_{1}Y^{(0)}_{\Gamma}+I_{\Phi\times\Phi}\nu-C^{T}P^{+}_{1}C\nu \right\}
\end{gather*}

або

\begin{gather*}
    U^{(0)}_{\Gamma}\in\Omega_{U^{(0)}_{\Gamma}}=
    \left\{ U^{(0)}_{\Gamma}:U^{(0)}_{\Gamma}=P^{+}_{2}C^{T}Y^{(0)}_{\Gamma}+I_{\Phi\times\Phi}\nu-P^{+}_{2}C^{T}C\nu \right\},
\end{gather*}

де $I_{\Phi\times\Phi}$ -- одинична матриця розміру $\Phi\times\Phi$, $\nu$-- довільний вектор розміру $\Phi$,
$P^{+}_{1}$, $P^{+}_{2}$ -- матриці псевдообернені до $P_{1}$, $P_{2}$.

Похибка моделювання $\left[ \eta^{(0)} \right]^{2}$ дорівнює

\begin{gather*}
    \left[ \eta^{(0)} \right]^{2}=\left[ Y^{(0)}_{\Gamma} \right]^T Z\left( C^{T} \right)Y^{(0)}_{\Gamma},
    Z(C^{T})=I_{\Phi\times\Phi}-C\left( C^{T}C \right)^{+}C^{T}.
\end{gather*}

В загальному випадку, зводячи систему

\begin{gather*}
    \left\{
    \begin{alignedat}{2}
    L(\partial^{2}_{x}, \partial_{t})y^{(q)}(s)=-P_{q}
    \left[
        \underbrace{
            ,\dots,
            y^{(q-1)}(s), \partial_x y^{(q-1)}(s), \partial_x^2 y^{(q-1)}(s)
        }_{3q}
        \right],
    s\in S^{T}_{0}, \\
    L^{\Gamma}(\partial_{x})y^{(q)}(s)=0, \in\bigcup\limits_{\rho=1}^{R_{\Gamma}} \Gamma_{0\rho}\times T_{0},
    \end{alignedat}
    \right.
\end{gather*}

до інтегрального вигляду та виконуючи дискретизацію, отримаємо

\begin{gather*}
    CU^{(q)}_{\Gamma} = Y^{(q)}_{\Gamma},\\
    U^{(q)}_{\Gamma}=\left(str\left( u^{(q)}_{\Gamma}
    (s'_{\varphi})\bigg|_{s'_{\varphi}\in S^{T}_{\infty 0}} \right) \varphi=1,\dots,\Phi \right)^{T}\\
    Y^{(q)}_{\Gamma}=col\left(Y^{\Gamma}(s_{w})
    -\int\limits_{S^{\infty}_{\infty}}L^{\Gamma}(\partial_{x})G(s_{w}-s')\bigg|_{s\in\Gamma^{T}_{0}}u(s')ds',
    s_{w}\in\bigcup\limits_{\rho=1}^{R_{\Gamma}} \Gamma_{0\rho}\times T_{0}
    \right)\\
    w=1,\dots,W_{\Gamma}\\
    C=col\left( str\left(L^{\Gamma}(\partial_{x}) G(s_{w}-s'_{\varphi})
    \bigg|_{\substack{
        s_{w}\in
        \bigcup\limits_{\rho=1}^{R_{\Gamma}} \Gamma_{0\rho}\times T_{0}\\
        s'_{\varphi}\in S^{T}_{\infty 0}
    }}
    \right)\varphi=1,\dots,\Phi \right)\\
    w=1,\dots,W_{\Gamma}
\end{gather*}

Маємо розв'язок системи

\begin{gather*}
    U^{(q)}_{\Gamma}\in\Omega_{U^{(q)}_{\Gamma}}=
    \left\{ U^{(q)}_{\Gamma}:U^{(q)}_{\Gamma}=C^{T}P^{+}_{1}Y^{(q)}_{\Gamma}+I_{\Phi\times\Phi}\nu-C^{T}P^{+}_{1}C\nu \right\}
\end{gather*}

або

\begin{gather*}
    U^{(q)}_{\Gamma}\in\Omega_{U^{(q)}_{\Gamma}}=
    \left\{ U^{(q)}_{\Gamma}:U^{(q)}_{\Gamma}=P^{+}_{2}C^{T}Y^{(q)}_{\Gamma}+I_{\Phi\times\Phi}\nu-P^{+}_{2}C^{T}C\nu \right\},
\end{gather*}

де $I_{\Phi\times\Phi}$ -- одинична матриця розміру $\Phi\times\Phi$, $\nu$-- довільний вектор розміру $\Phi$,
$P^{+}_{1}$, $P^{+}_{2}$ -- матриці псевдообернені до $P_{1}$, $P_{2}$.

Похибка моделювання $\left[ \eta^{(q)} \right]^{2}$ дорівнює

\begin{gather*}
    \left[ \eta^{(q)} \right]^{2}=\left[ Y^{(q)}_{\Gamma} \right]^T Z\left( C^{T} \right)Y^{(q)}_{\Gamma},
    Z(C^{T})=I_{\Phi\times\Phi}-C\left( C^{T}C \right)^{+}C^{T}.
\end{gather*}

\textbf{Висновки}.
Розглянуто нелінійну систему, яка функціонує у скінченій просторовій області та нескінченій
часовій області.
Отримано наближення нелінійної системи через нескінчену кількість лінійних систем.
Для кожної з лінійних систем побудовано функції Гріна.
Проведено моделювання розв'язків лінійних систем та отримано розв'язок початкової системи у вигляді ряду.
Знайдено похибку моделювання для нелінійних систем, що функціонують у скінченій просторовій області.

\section{Задача моделювання динаміки нелінійної в малому системи з розподіленими параметрами в обмежених
просторово-часових областях} \label{sec:sect2_4}

Нехай система ($\ref{eq:small_parameter_common_system_with_init_and_border_conditions}$) описується диференційним рівнянням з граничною  та початковою умовою

\begin{equation}
    \label{eq:small_parameter_common_system_with_init_and_border_conditions}
    \left\{
    \begin{alignedat}{2}
        L(\partial_x^2, \partial_t)y(s) + \varepsilon P(y(s), \partial_x y(s), \partial_x^2 y(s)) = u(s), s\in S_{0}^T, \\
        L^{0}(\partial_{t})y(x, 0) = Y^{0}(x), x\in\bigcup\limits_{r=1}^{R_{0}},t=0,\\
        L^{\Gamma}(\partial^{2}_{x})y(s) = Y^{\Gamma}(s),s\in\bigcup\limits_{\rho=1}^{R_{\Gamma}} \Gamma_{0\rho}\times T_{0},
    \end{alignedat}
    \right.
\end{equation}
де $s=(x,t)$, $x=(x_{1},\dots,x_{n})$, $S_{0}^{T}=S_{0}\times T_{0}$, $\partial_{x}=\left(
\frac{\partial}{\partial x_{1}}, \dots, \frac{\partial}{\partial x_{n}} \right)$,
$\partial^{2}_{x}=\left(\frac{\partial^{2}}{\partial x^{2}_{1}}, \dots, \frac{\partial^{2}}{\partial x^{2}_{n}} \right)$,
$L(\partial_x^2, \partial_t)$, $L^{\Gamma}(\partial^2_{x})$, $L^{0}(\partial_{t})$ -- лінійні диференціальні оператори з постійними коефіцієнтами при похідних,
$
L(\partial_x^2, \partial_t)=\sum\limits_{ \substack{i,j=1\\ i\neq j}}\left(l_{1i} \frac{\partial^2}{\partial x^2_i} +
l_{2i}\frac{\partial^2}{\partial x_{1i}
    \partial x_{1j}} + l_{3i} \frac{\partial}{\partial x_i} \right) + l_4 \frac{\partial}{\partial t}
$,
$
L^{0}(\partial_{t})=l^{0}_{1}\frac{\partial}{\partial t}
$,
$l^{0}_{1}\in\mathbb{R}$,
$
L^{\Gamma}(\partial_x^2)=\sum\limits_{ \substack{i,j=1\\ i\neq j}}\left(l^{\Gamma}_{1i} \frac{\partial^2}{\partial x^2_i} +
l^{\Gamma}_{2i}\frac{\partial^2}{\partial x_{1i}
    \partial x_{1j}} + l^{\Gamma}_{3i} \frac{\partial}{\partial x_i} \right)
$,
$l_{1i}$, $l_{2i}$, $l_{3i}$, $l_{4}\in\mathbb{R}$,
$l^{\Gamma}_{1i}$, $l^{\Gamma}_{2i}$, $l^{\Gamma}_{3i}\in\mathbb{R}$,
$P(y(s), \partial_x y(s), \partial_x^2 y(s))$ -- многочлен  своїх аргументів з постійними
коефіцієнтами при похідних,  $\varepsilon$ – малий параметр, $\varepsilon\in [0; 0.01]$,
$Y^{\Gamma}(x)\in\mathbb{C}^n$ -- задана функція, визначена в
$\bigcup\limits_{\rho=1}^{R_{\Gamma}}\Gamma_{0\rho}\times T_{0}$;
$\Gamma_{0\rho}\subseteq \Gamma_{0}$; $\Gamma_{0\rho}\cap\Gamma_{0\sigma} = \varnothing$;
$\rho\neq\sigma$; $\rho,\sigma = 1,\dots,R_{\Gamma}$, $R_{\Gamma}\in\mathbb{N}$; $u(s)\in\mathbb{C}^n$ --
задана функція, визначена в $S_{0}^{T}$, $S\subset S_{\infty}=\mathbb{R}^{n}$,
$S_{0}$ -- зв'язна замкнена множина; $\Gamma$ -- границя області $S_{0}$, $k,n\in\mathbb{N}$, $T_{0}=[0;T]$,
$T\in\mathbb{R}_{+}$.

Використовуючи розглянуту вище методику, запишемо систему рівнянь
\begin{gather*}
    \left\{
        \begin{alignedat}{2}
            L(\partial_x^2, \partial_t)y^{(0)}(s) - u(s) = 0, s\in S_{0}^T, \\
            L^{0}(\partial_{t})y^{(0)}(x, 0) = Y^{0}(x), x\in\bigcup\limits_{r=1}^{R_{0}},t=0,\\
            L^{\Gamma}(\partial^{2}_{x})y^{(0)}(s) = Y^{\Gamma}(s),s\in\bigcup\limits_{\rho=1}^{R_{\Gamma}} \Gamma_{0\rho}\times T_{0},
        \end{alignedat}
    \right.\\
    \dots\\
    \left\{
        \begin{alignedat}{2}
            L(\partial_x^2, \partial_t)y^{(1)}(s) = - P_{1}\left[
                y^{(0)}(s), \partial_x y^{(0)}(s), \partial_x^2 y^{(0)}(s)
            \right], s\in S_{0}^T, \\
            L^{0}(\partial_{t})y^{(1)}(x, 0) = 0, x\in\bigcup\limits_{r=1}^{R_{0}},t=0,\\
            L^{\Gamma}(\partial^{2}_{x})y^{(1)}(s) = 0,s\in\bigcup\limits_{\rho=1}^{R_{\Gamma}} \Gamma_{0\rho}\times T_{0},
        \end{alignedat}
    \right.\\
    \dots\\
    \left\{
        \begin{alignedat}{2}
            L(\partial_x^2, \partial_t)y^{(q)}(s) = \\
            = - P_{q}\left[
                         \underbrace{
                             ,\dots,
                             y^{(q-1)}(s), \partial_x y^{(q-1)}(s), \partial_x^2 y^{(q-1)}(s)
                         }_{3q}
            \right], s\in S_{0}^T, \\
            L^{0}(\partial_{t})y^{(q)}(x, 0) = 0, x\in\bigcup\limits_{r=1}^{R_{0}},t=0,\\
            L^{\Gamma}(\partial^{2}_{x})y^{(q)}(s) = 0,s\in\bigcup\limits_{\rho=1}^{R_{\Gamma}} \Gamma_{0\rho}\times T_{0},
        \end{alignedat}
    \right.
\end{gather*}


Розглянемо систему ($\ref{eq:small_parameter_common_system_with_init_and_border_conditions_iteration_0}$)

\begin{equation}
    \label{eq:small_parameter_common_system_with_init_and_border_conditions_iteration_0}
    \left\{
        \begin{alignedat}{2}
            L(\partial_x^2, \partial_t)y^{(0)}(s) - u(s) = 0, s\in S_{0}^T, \\
            L^{0}(\partial_{t})y^{(0)}(x, 0) = Y^{0}(x), x\in\bigcup\limits_{r=1}^{R_{0}},t=0,\\
            L^{\Gamma}(\partial^{2}_{x})y^{(0)}(s) = Y^{\Gamma}(s),s\in\bigcup\limits_{\rho=1}^{R_{\Gamma}} \Gamma_{0\rho}\times T_{0},
        \end{alignedat}
    \right.
\end{equation}

Нехай $y^{(0)}(s) = y^{(0)}_{\infty}(s) + y^{(0)}_{0}(s) + y^{(0)}_{\Gamma}(s)$, де $y^{(0)}_{0}(s)$, $y^{(0)}_{\Gamma}(s)$ --
функції, які можемо виразити через фіктивні динамічні збурення
$u^{(0)}_{0}(s')$,
$s=(x', t')$,
$s'\in S_{\infty}^{\infty 0} = S_{\infty} \times (T_{\infty}\textbackslash T_{0})$,
$y_{0}^{(0)}(s)=\int\limits_{S_{\infty}^{\infty 0}}G(s-s')\bigg|_{s\in S_{\infty}^{T}}u_{0}^{(0)}(s')ds'$,
$y_{\infty}^{(0)}(s)=\int\limits_{S_{\infty}^{\infty}}G(s-s')\bigg|_{s\in S_{\infty}^{T}}u_{0}^{(0)}(s')ds'$,
$ds'=dx'dt'$,
та
$u^{(0)}_{\Gamma}(s')$, $s'=(x', t')$, $s'\in S^{T}_{\infty 0}=(S_{\infty} \textbackslash S_{0})\times T_{0}$,
$y^{(0)}_{\Gamma}(s) = \int\limits_{S^{T}_{\infty 0}}G(s-s')\bigg|_{s\in S^{T}_{0}}u^{(0)}_{\Gamma}(s')ds'$,
$y^{(0)}_{\infty}(s) = \int\limits_{S^{\infty}_{\infty}}G(s-s')\bigg|_{s\in S^{T}_{0}}u^{(0)}_{\Gamma}(s')ds'$,
де $G(s - s')$ -- функція Гріна для необмеженій просторово-часовій області, для рівняння $L(\partial_x^2, \partial_t)y^{(0)}(s) - u(s) = 0$.

Отримаємо систему інтегральних рівнянь відносно $u^{(0)}_{0}(s')$ та $u^{(0)}_{\Gamma}(s')$
\begin{gather*}
    \left\{
    \begin{alignedat}{2}
        \int\limits_{S_{\infty}^{\infty 0}}L^{0}(\partial_{t})G(s-s')u_{0}^{(0)}(s')ds' =
        Y^{0}(x)-\int\limits_{S_{\infty}^{\infty}} L^{0}(\partial_{t})G(s-s')u_{0}^{(0)}(s')ds',\\
        x\in\bigcup\limits_{r=1}^{R_{0}} S_{0r},t=0.\\
        \int\limits_{S^{\infty}_{\infty}}L^{\Gamma}(\partial_{x})G(s-s')u(s')ds' +\\
        + \int\limits_{S^{T}_{\infty 0}}G(s-s')L^{\Gamma}(\partial_{x})u^{(0)}_{\Gamma}(s')ds' = Y^{\Gamma}(s),
        s\in\bigcup\limits_{\rho=1}^{R_{\Gamma}} \Gamma_{0\rho}\times T_{0}.
    \end{alignedat}
    \right.
\end{gather*}



Виконаємо дискретизацію тільки по змінній $s$.Отримаємо











\textbf{Висновки}.
Розглянуто нелінійну систему, яка функціонує у скінченій просторо-часовій області.
Отримано наближення нелінійної системи через нескінчену кількість лінійних систем.
Для кожної з лінійних систем побудовано функції Гріна.
Проведено моделювання розв'язків лінійних систем та отримано розв'язок початкової системи у вигляді
ряду.
Знайдено похибку моделювання для нелінійних систем, що функціонують у скінченій просторово-часовій області.


\section*{Висновки до другого розділу}
\addcontentsline{toc}{section}{Висновки до другого розділу}

Розглянуто нелінійні системи, які функціонують у нескінченій області, в обмеженій часовій області, обмеженій
просторовій області та скінченій просторово-часовій області.

Отримано наближення нелінійної системи через нескінчену кількість лінійних систем.

Запропонована методика моделювання дії початково-крайових умов за допомогою фіктивних функцій, які визначені за
межами області пошуку розв'язків.

Побудовані множини що моделюють функцій для різних видів дискретизації початково-крайових умов, а також для
випадку обмеженій тільки в часовій області та обмеженій тільки в просторовій області.

Для кожної з лінійних систем, які моделюють стан нелінійної системи, побудовано функції Гріна.

Проведено моделювання розв'язків лінійних систем та отримано розв'язок початкової системи у вигляді ряду.

Знайдено похибку моделювання для нелінійних систем, що функціонують у скінченій просторово-часовій області.
