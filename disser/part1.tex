\chapter{Огляд літератури та вибір методики досліджень} \label{chapt1}

В розділі проведено огляд літератури за обраною тематикою дисертаційного дослідження. Описана методика розв’язку
нелінійних задач з малим параметром, через приведення їх до системи лінійних задач (матричних інтегральних та
функціональних рівнянь), моделюючих нелінійне збурення. Приведена методика середньоквадратичного обернення зазначених
інтегральних та функціональних рівнянь.

Визначені предмет та методика дослідження, яка буде використана нами у подальших розділах роботи.

\section{Методика розв’язання  прямих та обернених просторових задач} \label{sect1_1}

Сучасна наука і техніка вимагають  все більш точних і надійних результатів що до прогнозів явищ, незалежно від того,
до якої галузі народного господарства вони належать. Збільшення точності розрахунків – невід’ємна потреба практики
як інженерної, так і суто наукової.

Серед всього різноманіття практичних задач, які постають перед науковцями-дослідниками задачі тепло- масопереносу
посідають окреме місце. Відомо, що процеси переносу тепла і маси (а вони є взаємозалежними) присутні у явищах, що
мають надважливе значення для народного господарства: робота ядерного реактора, взаємодія вузлів машин та механізмів,
виплавка сталі, теплове забруднення ґрунтів та інші.

Одним з методів дослідження процесів тепло- масопереносу є метод математичного моделювання. Під математичним
моделювання ми будемо розуміти розв’язок та дослідження властивостей системи рівнянь (алгебраїчної, диференціальної,
інтегральної або інтегро- диференціальної), що доповнена умовами однозначності (початковими та початково-крайовими),
які звужують множину можливих розв’язків. Математичне моделювання, зокрема, має рад переваг над фізичним моделюванням
та прямим фізичним дослідженням: воно дозволяє дослідити явища в умовах недоступних для фізичного дослідження;
дає можливість провести безліч експериментів (комп’ютерних, наприклад) без суттєвого збільшення їх вартості
а також провести оптимізацію функціонування конструкцій та механізмів. Але основним питанням, яке виникає після
отримання результатів моделювання є питання достовірності цих результатів. Вони мають пройти ретельну перевірку і у
разі невідповідності законам природи –  математична модель або методика проведення розрахунків мають бути перероблені.
Тому питання розробки нових математичних моделей, їх оптимізація та дослідження властивостей, розробка нових методів
моделювання і розширення області застосування для вже відомих методів є актуальним на даний час.

Сутність математичного моделювання, на наш погляд, полягає у тому, що знаючи причини явища, можна встановити його
наслідки (так звані прямі задачі) або знаючи наслідки явища, визначити його причини (обернені задачі). В задачах тепло-
масопереносу наслідком, зазвичай є температура тіла, а причинами є джерела та стоки тепла, які знаходяться у
досліджуваному тілі, форма тіла, теплова взаємодія тіла з оточуючим середовищем та інше.

Загальновідомо, що процеси тепло- масопереносу можуть бути описані системою диференціальних рівнянь (або одним
рівнянням) у частинних похідних та початково-крайовими умовами. Зазначимо, що математичні моделі  прямих та обернених
задач теплопереносу  ідентичні математичним моделям процесів переносу в акустиці \cite{Zaharov-Inversnaya}, гідродинаміці \cite{Grekov-Opredelenie},
фільтрації \cite{Ahmedzyanova-IssledovanieSposobov}, дифузії \cite{Gamayunov-Metod}.

Найбільш близьку до реальної картини дають розрахунки проведені не тільки з урахуванням залежності від координат
часу та простору величин, що входять до математичної моделі (характеристики речовин, коефіцієнти в граничних умовах,
внутрішніх та зовнішніх джерел тепла), а із урахуванням їх нелінійності \cite{Kozdoba-Reshenie}. Під лінеаризацією ми будемо розуміти
заміну нелінійної моделі на лінійну.  Похибкою лінеаризації назвемо різницю між розв’язком нелінійної та лінійної
задачами. У багатьох випадках лінеаризація виправдана. Але не зважаючи на те, що можна мінімізувати похибки
лінеаризації, часто точність розв’язків таких задач незадовільна, оскільки похибки лінеаризації виходять за
допустимі границі. Можна вважати допустимою максимальну похибку $\pm1\% $ від максимальної температури \cite{Kozdoba-Reshenie}. Похибка $\pm3 \div 5\% $
виправдана у деяких тестових розрахунках. Але похибка, що дорівнює $\pm10\% $ не припустима. Якщо максимальна робоча
температура дорівнює $ 1000 \degree C $, похибка буде $ 100 \degree C $. Відомо \cite{Nayfe-Vvedenie}, що границя текучості, границя міцності, границя тривалої
міцності та інші механічні характеристики різко змінюються кожні $ 10 \degree C $ при температурах порядку $ 1000 \degree C $. Отже похибка $ 100 \degree C $ є
абсолютно неприпустимою. Тому інтерес к дослідженню нелінійних математичних моделей задач нестаціонарної
теплопровідності є зрозумілим.

Розглянемо методи розв’язку прямих задач теплопровідності. Згідно \cite{Kozdoba-Reshenie} їх можна поділити прямі,непрямі, методі
теорії функції комплексного змінного, інші аналітичні методи. Можна виділити методи для розв’язку суто лінійних задач:
розділення змінних (метод Фур’є ), метод функції Гріна \cite{Stoyan-DoPobudovy}, метод теплових потенціалів, інтегральних перетворень.
В залежності від ядра інтегрального перетворення їх ділять на метод Лапласа (одностороннє, двостороннє перетворення),
Фур’є (пряме, обернене, косинус, синус, комплексне та узагальнене перетворення), Меллина, Бесселя (Ханкеля, Мейера,
Канторровича-Лебедєва) та інші. Для розв’язку нелінійних задач використовуються методи: варіаційні (Рітца, Гальоркіна,
Лейбензона, Трефтца, Канторовича, Біо, Гріна), методи зважених вичетів (коллокацій, метод Дірака, метод Гальоркіна,
метод моментів), методи ітерацій, методи зведення до інтегральних і диференціальних рівнянь різних типів,
метод кінцевих різниць та метод кінцевих елементів.

Обернені задачі тепло- масопереносу є частиною більш загальних задач оптимального керування та оптимального
проектування \cite{Kozdoba-Metody}. У технічній теплофізиці за відомими полями температур визначають початково-граничні умови,
теплофізичні характеристики тіл (коефіцієнти теплопровідності, теплоємності тощо), час досягнення заданих температур
та їх координати тощо. Перелічені характеристики є початковими даними для математичного моделювання прямих задач.
Вимоги високої точності розв’язків, без якої неможлива оптимізація по теплових характеристиках стали причиною
підвищеного інтересу до розв’язків обернених задач.Розв’язки обернених задач теплопереносу  дозволяють отримати
граничні умови, теплофізичні характеристики та інші умови однозначності тільки в лабораторних опитах, а, крім того,
у виробничому експерименті на експлуатованих об’єктах.

Згідно\cite{KozdobaKruckovskiy-MetodyResheniya} методи розв’язку обернених задач можна поділити на два класи: екстремальні та прямі. Критерієм розподілу
є наявність у методі операції пошуку екстремуму – мінімуму функціоналу нев’язок (відхилення модельних та
експериментальних температур у певній точці часу та простору). Прямими  у прикладній математиці прийнято називати
методи, що базуються на зведенні початкової задачі до розв’язку системи лінійних або нелінійних алгебраїчних
рівнянь.Екстремальні методи можна умовно поділити на дві групи: методи підбору та методи мінімізації. Серед прямих
методів можна також виділити дві групи методів: обернення розв’язків та обернення моделей. Ідея метода обернення
моделі є утому, що над математичною моделлю виконується ряд операцій для того щоб „прямо” отримати значення керуючої
функції (задача керування), початкових умов (задача відновлення) та інші параметри. Очевидно, що будь-яке
перетворення  математичної моделі, зокрема пов’язане з втратою точності (спрощення форми, часткова, або повна
лінеаризація, введення еквівалентних коефіцієнтів) повинно бути кількісно та якісно обґрунтоване. Зазначимо, що
існує багато методів перетворення моделей: підстановка Кірхгофа, заміна початкового диференціального рівняння
другого порядку у рівняння -ного порядку та його подальша лінеаризація, інтегральні перетворення та інші.

Одним з методів розв’язку обернених просторових задач є метод квазіобернення \cite{Lattes-MetodKvazi, Lions-NekotoryeMedody, Lions-Upravlenie, Lions-Optimalnoe}.
Основна ідея полягає в
змінені операторів, які входять до початкової задачі. В оператор вводяться додаткові диференціальні члени, які
вироджуються на границі, а також можуть бути спрямовані до нуля (тобто досить малі у певному сенсі). Умова
вродженості на границі області пошуку розв’язку, необхідна для того, щоб усунути складні граничні умови, які можуть
виникнути при введені додаткових диференціальних членів. Зазвичай, оператори отримані таким чином мають більш високий
порядок.

Очевидно, що існує безліч способів, якими можна змінити початковий диференціальний оператор. Загальним для них є те,
що некоректний у певному сенсі оператор замінюється близьким у певному розумінні оператором, але коректним. У випадку
задач теплопровідності ми можемо замінити початковий оператор загальним еволюційним параболічним оператором будь-якого
порядку; виконати заміну на інтегро-диференціальний оператор або параболічно-гіперболічну систему \cite{Lavrentiev-Lineynye}.

Ми також можемо поставити задачу про пошук наближення до температури в «середньому» на деякому інтервалі часу і
потім виконати заміну початкового диференціального оператора.

Ідея методу квазіобернення не нова і існує велика кількість методів подібного типу: метод псевдов’язкості \cite{Neuman-AMethod}, а
також його модифікації \cite{Oleynik-Razryvye}, \cite{Lax-OnTheStability}, метод еліптичної регулярізації
\cite{Oleynik-ObOdnoy}, метод регулярного виродження \cite{Friedrichs-Asymptotic}, метод
функцій штрафу  у варіаційних задачах \cite{Courant-Variational}, метод регулярізації по Тихонову
\cite{Tihonov-Metody, Tihonov-Nekorrektnye, Tihonov-Reguliariziruiuschie, Tihonov-Matematicheskoe, Tihonov-Chislennye},
 метод так званої допоміжної області \cite{Sayliev-Oreshenii}.

Важливим частковим випадком обернених задач математичної фізики є одномірні обернені задачі \cite{Lavrentiev-Odnomermye}. Вони являються
модельними для інтерпретації даних сейсморозвідки та електророзвідки. Існують наступні методи їх розв’язку: метод
неповного розділення змінних \cite{Lavrentiev-Odnomermye}, метод побудови «в цілому» розв’язків оберненої задачі з «розміщеним» джерелом,
метод розв’язку через спектральну функцію розподілу \cite{Gelfand-ObOpredelenii, Lavrentiev-Odnomermye}.
Отже, одномірні обернені задачі є зручним апаратом
розробки та перевірки нових методик пошуку розв’язку більш складних задач моделювання динаміки систем з розподіленими
параметрами.

Інша група методів розв’язку обернених задач – екстремальні \cite{Kozdoba-Metody}. В них для пошуку значень керуючої функції (задача
керування), початкових умов (задача відновлення) та інших параметрів використовується, як вже зазначалося вище,
принцип пошуку екстремуму (мінімуму) функціоналу нев’язок. Умовно екстремальні методи можна поділити на методи
підбору \cite{Arsenin-NekorrektnoPostavlennye}, методи мінімізації \cite{Polak-Chislennye}, методи
ідентифікації \cite{Stoyan-OZadache} і методи оптимального керування \cite{Kirichenko-Obshee}.

\textbf{Висновки}. Показано, що існує велика кількість методів розв’язку як прямих так і обернених задач теплопровідності.
Існує можливість підібрати той чи інший метод  і провести моделювання широкого класу задач.  Але не існує загальної
методики моделювання прямих та обернених нелінійних задач хоча б для певного вузького класу нелінійностей. Також не
існує загальної методики отримання оцінок точностей для знайдених  розв’язків.

\section{Метод збурень та методика розв’язання квазілінійних задач з суттєвими нелінійностями} \label{sect1_2}
\subsection{Метод збурень (метод малого параметра)} \label{sect1_2_1}

Більшість задач, з якими стикаються сьогодні фізики та спеціалісти з прикладної математики не має точних розв’язків.
Основними причинами цього є змінні коефіцієнти при операторах диференціювання, нелінійні граничні умови на невідомих
границях складної форми або нелінійні рівняння руху. Щоб отримати розв’язок таких задач ми змушені використовувати
різні наближення або чисельні методи. Основним з наближених методів є метод збурень
\cite{Van-Dayk-Metody, Coul-Metody, Mitropolskiy-Problemy, Mitropolskiy-Leksii, Nayfe-Vvedenie, Erdein-Asimptoticheskie}.
Цей метод є одним з ефективних засобів сучасної прикладної математики. Він дозволяє отримати наближені аналітичні
та чисельні розв’язки складних нелінійних та лінійних крайових задач. Він використовується як для звичайних
диференціальних рівнянь так і для рівнянь у частинних похідних.

Ідея метода вперше виникла при розв’язку задач небесної механіки. Під „збуренням”  там розуміють вплив небесних
тіл на орбітальний рух планет. При цьому вплив факторів, що вносять збурення не приводить до суттєвої зміни
задачі, а лише обумовлює деякі зміни, які можуть бути враховані у першому та подальших наближеннях.

Основа метода збурень полягає в використанні малості деяких величин, які входять  у задачу, в порівнянні з
величинами, що входять у початкову математичну модель досліджуваного явища
\cite{Aizen-KvoprosuPrimeneniya, Aizen-ReshenieNelineynoyZadachi, Baylens-ReshenieMetodom, Van-Dayk-Metody,
KozdobaChumakov-MetodyPrimeneniya, KozdobaChumakov-ReshenieVosmushenien, KozdobaChumakov-ReshenieMalymParametrom,
Nayfe-Vvedenie, Novikov-sintez, Olson-Primenenie, Furman-Reshenie, Erdein-Asimptoticheskie}.
У відповідності до метода збурень розв’язок задачі представляється декількома (зазвичай двома) першими членами
асимптотичного ряду. У багатьох випадках питання про збіжність цих рядів залишається відкритим.
Але для кількісного та якісного аналізу цей розв’язок може бути більш корисний, ніж ряди що збігаються абсолютно
\cite{Nayfe-Vvedenie}.

Метод збурень можна застосовувати до задач, що мають близькі задачі з точним (або знайденим з певною точністю)
розв’язком. Як буде показано нижче, ця „наближеність” однієї задачі до іншої є досить тонким моментом і може
бути створена штучно. Наприклад, початкова нелінійна задача зводиться до такої, що близька до задачі,
яка має достатньо точний розв’язок. Потім необхідно провести вибір та перетворення малого параметра або штучне
його введення таким чином, щоб складність розв’язку отриманої задачі була мінімальна \cite{KozdobaChumakov-MetodyPrimeneniya}.

Ми будемо називати збурення поверхневими, якщо вони відносяться до граничних умов, та — об’ємними, якщо вони
відповідають збуренню основного диференціального  рівняння. У загальному випадку збурення — це
будь яке відхилення від задачі, яка має  точний або наближений розв’язок \cite{Mors-metody}.
Відхилення вважається малим і тому розв’язок можна розкласти в ряд по степеням параметра.

Розглянемо диференціальне рівняння, яке містить деякий параметр $\varepsilon$.

\begin{equation}
  \label{eq:eq_epsylon}
\begin{array}{rcl}
D(\partial_x, \partial_x^2, \partial_t, y(s), \varepsilon) & = & 0, s\in S_0^T, \\
L^\Gamma(\partial_x, \partial_x^2)y(s) & = & Y^\Gamma(s), s \in \Gamma_0^T, \\
L^0(\partial_t)y(x, 0) & = & Y^0(x), x \in S_0, t = 0,
\end{array}
\end{equation}

де $ D(\partial_x, \partial_x^2, \partial_t, y(s), \varepsilon) $ – нелінійний диференціальний оператор у частинних
похідних з постійними коефіцієнтами при похідних, $ L^\Gamma(\partial_x, \partial_x^2) $, $ L^0(\partial_t) $ –
лінійні диференціальні оператори з постійними коефіцієнтами при похідних,
$ \partial_x = \left( \frac{\partial}{\partial x_1},\cdots , \frac{\partial}{\partial x_n}\right) $,
$ \partial_x^2 = \left( \frac{\partial^2}{\partial x_1^2},\cdots , \frac{\partial^2}{\partial x_n^2}\right) $,
$ \partial_t =  \frac{\partial}{\partial t} $,
$ Y^\Gamma(s) $, $Y^0(x) \in C ^n $, $\varepsilon$ – малий параметр,
$ S_0^T = S_0 \times T_0$, $ S_0 \subset S_\infty = \left(-\infty < x_i < \infty \right) $,
$ T_0 = \left[0; T\right] $, $ T \in \mathbb{R}_+ $,
$ s=(x,t) $, $ \Gamma_0^T = \Gamma_0 \times T_0 $, $ \Gamma_0 $  – границя області $ S_0 $.
Припустимо, що при деякому $\varepsilon = \varepsilon_0$ нам вдалося знайти частинний розв’язок рівняння (\ref{eq:eq_epsylon}). Тоді ми можемо записати

\begin{equation}
  \label{eq:eq_sol_epsylon}
  y(s) = y_0(s) + (\varepsilon - \varepsilon_0)y_1(s) + (\varepsilon - \varepsilon_0)^2y_2(s) + \cdots,
\end{equation}

де $y_1(s)$, $y_2(s)$, $\ldots$ – функції, що необхідно визначити.

Бувають випадки, коли рівняння (\ref{eq:eq_epsylon}) не містить параметра.
Тобто \\ $D(\partial_x, \partial_x^2, \partial_t, y(s), \varepsilon) = 0$, $s\in S_0^T$.
Тоді його вводять штучно \cite{Kozdoba-Reshenie}, таким чином, щоб при деякому окремому значенні параметра
$\varepsilon = \varepsilon_0'$ була справедливою тотожність

$$
D(\partial_x, \partial_x^2, \partial_t, y(s), \varepsilon_0') = D(\partial_x, \partial_x^2, \partial_t, y(s)),
$$

а при іншому значенні $\varepsilon = \varepsilon_0''$ розв’язок $y_0(s)$ рівняння
$D(\partial_x, \partial_x^2, \partial_t, y(s), \varepsilon_0'') = 0 $, $s \in S_\infty^\infty $ був би відомий.
В цьому разі ми шукаємо розв’язок у вигляді ряду

\begin{equation}
  \label{eq:solution_as_riad}
y(s) = y_0(s) + \varepsilon y_1(s) + \varepsilon^2 y_2(s) + \cdots,
\end{equation}

після чого даємо значення $ \varepsilon = \varepsilon_0' $.

Якщо початкове рівняння можна представити у вигляді

$$
D(\partial_x, \partial_x^2, \partial_t, y(s)) \equiv L(\partial_x, \partial_x^2, \partial_t) +
F(\partial_x, \partial_x^2, \partial_t, y(s)) = 0, s\in S_0^T
$$

де $L(\partial_x, \partial_x^2, \partial_t)$ – лінійний оператор, $F(\partial_x, \partial_x^2, \partial_t, y(s))$
– нелінійний член, який можна розглядати як збурення лінійного оператора $L(\partial_x, \partial_x^2, \partial_t)$,
$ S_0^T = S_0 \times T_0$, $ S_0 \subset S_\infty = \left(-\infty < x_i < \infty \right) $,
$ T_0 = \left[0; T\right] $, то введення малого параметра $\varepsilon$ можна провести наступним чином \cite{Kozdoba-Reshenie}

\begin{equation}
\label{eq:added_epsi}
D(\partial_x, \partial_x^2, \partial_t, y(s), \varepsilon) \equiv L(\partial_x, \partial_x^2, \partial_t) +
\varepsilon F(\partial_x, \partial_x^2, \partial_t, y(s)) = 0.
\end{equation}

Згідно до теореми про аналітичну залежність розв’язку від параметра \cite{Elsgoltz-Differentsialnye}
при достатньо малих по модулю значень $\varepsilon$
розв’язок $ y(s)$ рівняння (\ref{eq:added_epsi}) буде аналітичною функцією від параметра $\varepsilon$.
При цьому функція $ F(\partial_x, \partial_x^2, \partial_t, y(s))$ повинна бути
неперервною та аналітично залежати від $ y(s)$ та її похідних в області $ s\in S_0^T$. У цьому разі розв’язок можемо
шукати у вигляді (\ref{eq:solution_as_riad}), після чого припускаємо $\varepsilon = 1$ , та отримаємо кінцеву відповідь.

Підставимо розв’язок (\ref{eq:solution_as_riad}) в (\ref{eq:added_epsi}) та почленно продиференціюємо
його необхідну кількість разів. Використовуючи
теорему про добуток двох степеневих рядів
$\left(\sum_{i=0}^\infty \varepsilon^i y_i(s)\right)\left(\sum_{i=0}^\infty \varepsilon^i y_i(s)\right) =
\sum_{i=0}^\infty \varepsilon^i \sum_{j=0}^i y_{i-j}(s)y_j(s)$,
можемо згрупувати члени рівняння при однакових степенях $\varepsilon$. Оскільки
рівняння (\ref{eq:solution_as_riad}) справедливе для будь яких $\varepsilon$ та послідовність степенів  лінійно незалежна, то коефіцієнт при кожній
ступені обертається в нуль незалежно. Таким чином, можемо записати систему

$$
\left\{
\begin{array}{rcl}
L(\partial_x, \partial_x^2, \partial_t) y_0(s) & = & 0, s\in S_0^T, \\
L^\Gamma(\partial_x, \partial_x^2)y_0(s) & = & 0, s \in \Gamma_0^T,  \\
L^0(\partial_t)y_0(x, 0) & = & Y^0(x), x \in S_0, t = 0,
\\
\\
L(\partial_x, \partial_x^2, \partial_t) y_0(s) + F_1(\partial_x, \partial_x^2, \partial_t, y_0(s), y_1(s)) & = & 0, s\in S_0^T, \\
L^\Gamma(\partial_x, \partial_x^2)y_1(s) & = & 0, s \in \Gamma_0^T,  \\
L^0(\partial_t)y_1(x, 0) & = & Y^0(x), x \in S_0, t = 0,
\\
\hdotsfor{3}
\\
L(\partial_x, \partial_x^2, \partial_t) y_0(s) + F_q(\partial_x, \partial_x^2, \partial_t, y_0(s), y_1(s)), \ldots y_q(s)) & = & 0, s\in S_0^T, \\
L^\Gamma(\partial_x, \partial_x^2)y_1(s) & = & 0, s \in \Gamma_0^T,  \\
L^0(\partial_t)y_1(x, 0) & = & Y^0(x), x \in S_0, t = 0,
\end{array}
\right.
$$

де $q = 1, 2, 3\ldots$

Розв’язуючи послідовно ці рівняння, отримаємо $y_0(s)$, $y_1(s)$, $\ldots$

Зазвичай при використанні метода збурень у якості збурюючих факторів приймають одну або декілька координат простору і
часу, або один з безрозмірних параметрів. У цьому випадку кажуть про збурення координат або деякого числового
коефіцієнта, що входить в крайову задачу \cite{Van-Dayk-Metody, Kozdoba-Reshenie}. Природно, що в такому разі область застосування метода збурення
обмежується абсолютною величиною малого параметра і включає до себе лише задачі з несуттєвими нелінійностями.

Проте величиною збурення може бути функція або оператор. Ми можемо штучно вводити параметр $\varepsilon$, що суттєво розширяє
область застосування метода від задач, які близькі до лінійних до задач з суттєвими нелінйностями. Отже, методика
введення малого параметра представляє окремий науковий інтерес, і успішність її використання залежить як від
конкретної задачі так і від майстерності досліджувача. Цій проблематиці присвячена робота \cite{Marin-Issledovanie}.

У розглянутих вище задачах ми підставляли функціональні ряди типу (\ref{eq:eq_sol_epsylon}) та (\ref{eq:solution_as_riad})
в диференційне рівняння та виконували
додавання, добуток та диференціювання цих рядів. Відомо, що для асимптотичних рядів операції додавання та добутку
є допустимими. Але диференціювати ряди в загальному випадку не можна \cite{Erdein-Asimptoticheskie}. Правомірність почленного диференціювання
ряду та його збіжність неважко встановити, якщо відомий вираз для загального члену ряду. Проте знайти цей вираз
вкрай важко і, зазвичай, дослідити збіжність розв’язку нелінійної задачі, в загальному випадку не вдається.

Згідно \cite{Van-Dayk-Metody} було б помилкою вважати, що абсолютна збіжність є практично необхідною. Для прикладних задач нам
достатньо перших двох або трьох членів ряду, якщо вони в цілому досить швидко наближаються до розв’язку. Відомі,
навіть, випадки, коли ряди \cite{Van-Dayk-Metody}, що не мають збіжності дають кращі наближення, ніж ряди, що збігаються.

\textbf{Приклад 1.1.} Розглянемо, як приклад задачу одномірної стаціонарної теплопровідності в довгому стержні кругового
поперечного перерізу \cite{Coul-Metody}, форма якого задається рівнянням

$$
\nu(x_0, x_1) \equiv x_1 - b F \left( \frac{x_0}{a} \right) = 0, x_0 \in[0; a], a,b = const,
$$

% FIXME: check about R. Why R here if R is a set of rational numvers, not a function.

де $F \left( \frac{x_0}{a} \right) \in C^n$  – функція бокової поверхні.

Нехай, бокова поверхня стержня теплоізольована, $\frac{\partial y(x_0, x_1)}{\partial \overrightarrow{n}}$ , $\overrightarrow{n}$
– нормаль до бокової поверхні. На кінцях стержня температура задана у вигляді

$$
y(0, x_1) = Y^*Y_0^\Gamma \left( \frac{x_1}{b} \right), y(a, x_1) = Y^*Y_1^\Gamma \left( \frac{x_1}{b} \right)
$$

%FIXME: check boundary conditions

де $Y^* \in  \mathbb{R}$ – характерна температура, $Y^*Y_0^\Gamma \left( \frac{x_1}{b} \right)$,
$y(a, x_1) = Y^*Y_1^\Gamma \left( \frac{x_1}{b} \right)$, $Y^*Y_0^\Gamma, Y^*Y_1^\Gamma \in C^n$  – відомі функції. Будемо розглядати випадок,
коли $\left| \frac{b}{a} \right| \ll 1$. Стаціонарний
тепловий поток при постійних термічних властивостях матеріалу стержня описується  рівнянням Лапласу у випадку осевої симетрії

$$
\frac{\partial^2 y(x_0, x_1)}{\partial x_1^2} + \frac{1}{x_1} \frac{\partial y(x_0, x_1)}{\partial x_1} +
\frac{\partial^2 y(x_0, x_1)}{\partial x_0^2} = 0.
$$

Гранична умова, що виражає факт теплоізоляції бокової поверхні стержня може бути записана таким чином:
$\frac{\partial y(x_0, x_1)}{\partial x_1} - \frac{b}{a} \frac{d}{d x_0} F \left( \frac{x_0}{a} \right)
\frac{\partial y(x_0, x_1)}{\partial x_0} = 0$
 при $ x_1 - b F \left(\frac{x_0}{a} \right) = 0$, $x_0 \in [0; a]$. % FIXME: Why z depends on (x0, x1)
Виразимо задачу у безрозмірних координатах: $z_0 = \frac{x_0}{a}$, $z_1=\frac{x_1}{b}$,
$\Theta(z_0, z_1) = \frac{y(x_0,x_1)}{Y^*}$. Отже,

\begin{equation}
\label{eq:heat_theta}
\frac{\partial^2\Theta(z_0, z_1)}{\partial z_1^2} + \frac{1}{z_1} \frac{\partial \Theta(z_0, z_1)}{\partial z_1} +
 \varepsilon^2\frac{\partial^2\Theta(z_0, z_1)}{\partial z_0^2} = 0, (z_0, z_1) \in S_0
\end{equation}

\begin{equation}
\begin{array}{c}
\Theta(0, z_1) = Y_0^\Gamma(z_1), z_1 \in [0; F(0)],  \\
\Theta(1, z_1) = Y_1^\Gamma(z_1), z_1 \in [0; F(1)],
\end{array}
\end{equation}

\begin{equation}
\frac{\partial}{\partial z_1} \Theta(z_0, F(z_0)) - \varepsilon^2 \frac{d}{dz_0} F(z_0) \frac{\partial}{\partial}
\Theta(z_0, F(z_0)) = 0, z_0 \in [0; 1],
\end{equation}

де $S_0 = [0; 1] \times [0; F(z_0)] \forall z_0 \in [0; 1]$, $ \varepsilon = \frac{b}{a}$.

Припустимо, що ми маємо розв’язок при $\varepsilon \to 0$, який не залежить від $\varepsilon$. Нехай

%FIXME: why epislon^2 here right after epsilon^0, not an epsilon^1 in the second part?
\begin{equation}
\label{eq:theta_solution}
\Theta(z_0, z_1) = \Theta_0(z_0, z_1) + \varepsilon^2\Theta_1(z_0,z_1) + \cdots
\end{equation}

при умові, що цей розв’язок буде використаний вдалині від кінців стержня. Підставимо (\ref{eq:theta_solution}) в
(\ref{eq:heat_theta}). Групуючи члени
при однакових степенях $\varepsilon$, отримаємо системи рівнянь

\begin{equation}
\left\{
\begin{alignedat}{2}
&\frac{\partial^2\Theta_0(z_0, z_1)}{\partial z_1^2} + \frac{1}{z_1} \frac{\partial \Theta_0(z_0, z_1)}{\partial z_1}  =  0,
(z_0, z_1) \in S_0 \\
&\frac{\partial}{\partial z_1} \Theta_0(z_0, F(z_0))  =  0, z_0 \in [0; 1],
\end{alignedat}
\right.
\end{equation}

\begin{equation}
\label{eq:theta_2nd_iter}
\left\{
\begin{alignedat}{2}
&\frac{\partial^2\Theta_1(z_0, z_1)}{\partial z_1^2} + \frac{1}{z_1} \frac{\partial\Theta_1(z_0, z_1)}{\partial z_1}  =
-\frac{\partial^2\Theta_0(z_0, z_1)}{\partial z_0^2}, (z_0, z_1) \in S_0, \\
&\frac{\partial}{\partial z_1}\Theta_1(z_0, F(z_0)) - \frac{d}{dz_0}F(z_0)\frac{\partial}{\partial z_0}
\Theta_0(z_0, F(z_0))  =  0, z_0 \in [0; 1],
\end{alignedat}
\right.
\end{equation}

де $S_0 = [0; 1] \times [0; F(z_0)] \forall z_0 \in [0; 1]$.

Розглянемо першу систему. Її розв’язком буде

$$
\Theta_0(z_0, z_1) = A_0(z_0) + B_0(z_0) \ln z_1,
$$

де $A_0(z_0), B_0(z_0) \in C^n$  – довільні функції. Якщо вимагати, щоб температура на осі стержня була кінцевою,
то $B_0(z_0) = 0$.
Отже, $\Theta_0(z_0, z_1) = A_0(z_0)$. Тоді можемо підставити $\Theta_0(z_0, z_1)$  в (\ref{eq:theta_2nd_iter}). Отримаємо

\begin{equation}
\label{eq:theta_2nd_iter_new}
\left\{
\begin{alignedat}{2}
&\frac{\partial^2\Theta_1(z_0, z_1)}{\partial z_1^2} + \frac{1}{z_1} \frac{\partial\Theta_1(z_0, z_1)}{\partial z_1}  =
-\frac{d^2\Theta_0(z_0, z_1)}{dz_0^2}, (z_0, z_1) \in S_0, \\
&\frac{\partial}{\partial z_1}\Theta_1(z_0, F(z_0)) - \frac{d}{dz_0}F(z_0)\frac{d}{dz_0}
\Theta_0(z_0, F(z_0))  =  0, z_0 \in [0; 1],
\end{alignedat}
\right.
\end{equation}

де $S_0 = [0; 1] \times [0; F(z_0)] \forall z_0 \in [0; 1]$.

%FIXME: why not consider ln here?
Не враховуючи логарифмічний член у виразі $\Theta_0(z_0, z_1) = A(z_0) + B_0(z_0)\ln z_1$,
отримаємо розв’язок для системи (\ref{eq:theta_2nd_iter_new}) згідно \cite{Coul-Metody}:

$$
\Theta_1(z_0, z_1) = A_1(z_0) - \frac{z_1^2}{4}\frac{d^2 A_0(z_0)}{dz_0^2}.
$$

Аналогічно $A_1(z_0)$ знайдемо з рівняння для $\Theta_2(z_0, z_1)$. Підставляючи
$\Theta_1(z_0, z_1)$ в граничну умову, отримаємо

$$
-\frac{F(z_0)}{2}\frac{d^2A(z_0)}{dz_0^2} = \frac{d}{dz_0}F(z_0)\frac{d}{dz_0}A_0(z_0) \Leftrightarrow
\frac{d}{dz_0}\left(F(z_0)^2\frac{d}{dz_0}A_0(z_0)\right) = 0.
$$

Знайдемо граничні умови для рівняння (\ref{eq:heat_theta}) при $z_0\to0$ та $z_0\to1$.
В околі точки $z_0=0$ єдина характерна границя, яка дозволяє виконати
зрощування, зберігаючи структуру рівняння, є границя, для якої
$\tilde{z_0} = z_0 / \varepsilon = const$. Розглянемо асимптотичний розклад в околі точки $z_0=0$:

$$
\Theta(z_0, z_1, \varepsilon) = \vartheta_0(\tilde{z_0}, z_1) + \varepsilon^2 \vartheta_1(\tilde{z_0}, z_1) + \cdots,
$$

де $\tilde{z_0} = z_0 / \varepsilon$. Будемо розглядати лише перший член цього розкладання.
Тоді для $\vartheta_0(\tilde{z_0}, z_1)$ отримаємо рівняння

$$
\frac{\partial^2\vartheta_0(\tilde{z_0}, z_1)}{\partial z_1^2} +
\frac{1}{z_1}\frac{\partial\vartheta_0(\tilde{z_0}, z_1)}{\partial z_1} +
\frac{\partial^2\vartheta_0(\tilde{z_0}, z_1)}{\partial\tilde{z_0}^2} = 0.
$$

Запишемо граничну умову

$$
\frac{\partial}{\partial z_1}\vartheta_0(\tilde{z_0}, F(\varepsilon\tilde{z_0})) + \ldots =
\varepsilon^2\frac{d}{d\tilde{z_0}}F(\varepsilon\tilde{z_0})\frac{1}{\varepsilon}\frac{\partial}{\partial\tilde{z_0}}
\vartheta_0(\tilde{z_0}, F(\varepsilon \tilde{z_0})) + \ldots
$$

Отже, при $\varepsilon\to 0$ маємо

$$
\frac{\partial}{\partial z_1}\vartheta_0(\tilde{z_0}, F(0)) = 0.
$$

Якщо $\tilde{z_0} = 0$, то $\vartheta(0, F(0)) = Y_0^\Gamma(z_1)$, $z_1 \in [0; F(0)]$.
Таким чином, необхідно розв’язати задачу розрахунку теплового потоку у теплоізольованому  циліндрі.
Ми виконуємо зрощення, яке відповідає будь-якій проміжній границі $\tilde{z_0} \to \infty$, $z_0 \to 0$.
Умова зрощення має простий вигляд

$$
\lim_{\tilde{z_0} \to \infty} \vartheta_0(\tilde{z_0}, z_1) = A_0(0).
$$

Розв’язок задачі для $\vartheta_0(\tilde{z_0}, z_1)$ можна шукати методом розділення змінних у вигляді
добутку функцій $\vartheta_0(\tilde{z_0}, z_1) = \exp(-\lambda\tilde{z_0})J_0(\lambda z_1)$, $\lambda \geq 0$,
де $J_0(\lambda z_1)$ – функція Бесселя
першого роду. Тоді з рівняння $\frac{\partial}{\partial z_1}\vartheta(\tilde{z_0}, F(0)) = 0$ отримаємо
трансцендентне рівняння для власних значень $\lambda_n\frac{d}{d z_1} J_0(\lambda_0 z_1) = 0$. Це рівняння дає
нескінчену кількість коренів $\lambda_1, \lambda_2, \lambda_3, \ldots, (\lambda_0 = 0)$.
Їм відповідає нескінчена повна система власних функцій. Таким чином,

$$
\vartheta_0(\tilde{z_0}, z_1) = a_0 + \sum_{n = 1}^\infty a_n\exp(-\lambda_n \tilde{z_0}) J_0(\lambda z_1).
$$

Згідно \cite{Coul-Metody} із рівняння
$\frac{d}{d z_1} \left( z_1 \frac{d J_0(\lambda z_1)}{d z_1} \right) + \lambda^2 z_1 J_0(\lambda z_1) = 0$
для $J_0(\lambda z_1)$ з урахуванням того, що
$\lambda_n\frac{d}{d z_1} J_0(\lambda_n z_1) = 0$
інтегруванням від $0$ до $F(0)$ знайдемо, що $\int\limits_0^{F(0)}J_0(\lambda_n z_1) z_1 dz_1 = 0$. Отже,

$$
\int\limits_0^{F(0)}\vartheta_0(z_0, z_1)z_1 dz_1 = \frac{F(0)^2}{2}a_0  \Rightarrow
\int\limits_0^{F(0)}\vartheta_0(0, z_1)z_1 dz_1 = \frac{F(0)^2}{2}a_0  \Rightarrow
$$

$$
\Rightarrow\frac{2}{F(0)^2} \int\limits_0^{F(0)}Y^\Gamma_0(z_1)z_1 dz_1 = a_0.
$$

Решту коефіцієнтів можна знайти, використовуючи ортогональність власних функцій

$$
\int\limits_0^{F(0)}J_0(\lambda_n z_1)J_0(\lambda_m z_1)z_1 dz_1 = 0, n \neq m
$$

$$
\int\limits_0^{F(0)}J_0(\lambda_n z_1)J_0(\lambda_m z_1)z_1 dz_1 = \gamma_m^2, n = m,
$$

$$
m,n = 1, 2, \ldots
$$

Аналогічні розрахунки проведемо при $z_0 \to 1$. Отримаємо

$$
\frac{2}{F(1)^2}\int\limits_0^{F(1)}Y_1^\Gamma(z_1)z_1 dz_1 = b_0,
$$

$$
\int\limits_0^{F(1)}J_0(\mu_n z_1) J_0(\mu_m z_1) z_1 dz_1 = 0, n \neq m,
$$

$$
\int\limits_0^{F(1)}J_0(\mu_n z_1) J_0(\mu_m z_1) z_1 dz_1 = \gamma_m^2, n = m,
$$

$$
m,n = 1, 2, \ldots
$$

\textbf{Висновки}. Сфера застосування метода малого параметра є досить широкою. Він може бути присутнім в граничних
умовах, початкових умовах та у диференціальному рівнянні. Збурення може бути проведено по координатах часу і простору
та у диференціальному операторі у якості безрозмірних коефіцієнтів. Малий параметр може бути введений штучно, що
додатково розширює область застосування метода.

\subsection{Використання метода малого параметра для розв’язання задач з суттєвими нелінійностями} \label{sect1_2_2}

У роботах \cite{Ivanov-KResheniyu, LiOrlov-KTeorii, Marin-ObOdnom, Iampolski-OMetode, Lock-AStudy, Tadjbaksh-Free}
 метод малого параметра використовувався для розв’язку нелінійних
задач з несуттєвими нелінійностями. У цьому розділі ми покажемо, як можна розширити метод малого
параметра (метод збурень) на задачі зі суттєвими нелінійностями. У якості збурюючей функції
приймається відхилення від належним засобом апроксимованої нелінійності. При цьому ми можемо зняти
обмеження про те, що нелінійність обов’язково повинна бути слабкою (несуттєвою). Нелінійність може
вносити велике збурення в початкову задачу, але відхилення між дійсною та апроксимуючою нелінійностями буде
малим. Крім того можна застосувати метод малого параметра при всіх типах нелінійностей: в початково-крайових
умовах, у коефіцієнтах диференціального оператора та у правої частині початкового диференціального рівняння.

\textbf{Приклад 1.2}. Розглянемо наступну математичну модель загальної нелінійної задачі
теплопровідності, записану для тіл канонічної форми \cite{Kozdoba-Metody}:

$$
\frac{\partial}{\partial x} \left[\Lambda(y)\frac{\partial y(x,t)}{\partial x} \right]
+ \frac{K - 1}{x}\left[ \Lambda(y)\frac{\partial y(x,t)}{\partial x} \right]
- C(y) \frac{\partial y(x,t)}{\partial t} = u(x, t, y), (x, t) \in S_0^T,
$$

$$
\Lambda(y)\frac{\partial y(x, t)}{\partial x} = F\left[Y^\Sigma(t), Y^\Gamma(t) \right], x = 1, t\in T_0,
$$

$$
\frac{\partial y(x, t)}{\partial x} = 0, x = 0, t\in T_0,
$$

$$
y(x, t) = Y^0(x), t = 0,
$$

де $\Lambda(y), C(y), u(x,t, y), F[\cdot], Y^\Sigma(x, t), Y^\Gamma(x, t), Y^0(x) \in C^n$ – відомі функції;
$K$, яке дорівнює 1, 2 або 3 відповідає площині, циліндру або сфері; $S_0^T \in S_0 \times T_0$, $S_0=[0; 1],
T_0=[0; \infty)$.
Згідно \cite{Furman-Reshenie} будемо шукати розв’язок у вигляді $\nu(x, t) = F[Y^\Sigma(t), y]$.
Теплофізичні характеристики запишемо наступним чином

$$
C(y) = [a_1 + \varphi(y)] \frac{\partial F[\cdot]}{\partial y},
\Lambda(y) = [a_2 + \Psi(y)] \frac{\partial F[\cdot]}{\partial y}
$$

де $a_1, a_2 = const$, $\varphi(y), \Psi(y) \in C^n$ – довільні функції. Введення функцій $\varphi(y)$ та $\Psi(y)$
дозволяє зменшити похибку апроксимації,
а також зняти обмеження $\frac{C(y)}{\Lambda(y)} = const$ що до функцій $\Lambda(y), C(y)$, у роботі \cite{Furman-Reshenie}. Проведемо диференціювання складної
функції $\nu(x, t) = F[Y^\Sigma(t), y]$ та отримаємо вирази для $\frac{\partial y(x, t)}{\partial x}$ і
$\frac{\partial y(x, t)}{\partial t}$.

%FIXME: check this sequation. may be mistake in last one
$$
\frac{\partial y(x, t)}{\partial x} = \frac{\partial\nu(x,t)}{\partial x}\left(\frac{\partial\nu(\cdot)}{\partial y} \right),
\frac{\partial y(x, t)}{\partial t} = \frac{\partial\nu(x,t)}{\partial t} - \frac{\partial\nu(x, t)}{\partial Y^\Sigma} \frac{dY^\Sigma}{dt}.
$$

З урахуванням цих виразів отримаємо систему для нової функції $\nu(x, t)$

\begin{equation}
\label{eq:major_non_lin_p1}
\begin{multlined}
\frac{\partial}{\partial x}\left\{[a_2 + \Psi^*(\nu)]\frac{\partial\nu(x, t)}{\partial x}\right\} +
\frac{K - 1}{x}\left\{[a_2 + \Psi^*(\nu)]\frac{\partial\nu(x, t)}{\partial x} \right\} - \\
- [a_1 + \varphi^*(\nu)]\frac{\partial F}{\partial Y^\Sigma}\frac{\partial Y^\Sigma(x, t)}{\partial t} =
u^*(x, t, \nu), (x, t) \in S_0^T,
\end{multlined}
\end{equation}

\begin{equation}
\label{eq:major_non_lin_p2}
\begin{multlined}
\left\{[a_2 + \Psi^*(\nu)\frac{\partial\nu(x, t)}{\partial x}] \right\} = \nu(1, t), x = 1, t\in T_0, \\
\frac{\partial\nu(x, t)}{\partial x} = 0, x = 0, t\in T_0,
\end{multlined}
\end{equation}

\begin{equation}
\label{eq:major_non_lin_p3}
\nu(x, t) = Y^{*0}(x), t = 0,
\end{equation}

де $\Psi^*(\nu), \varphi^*(\nu), u^*(x,t, \nu), Y^{*0}(x) \in C^n$ – відомі функції.
Позначимо

$$\Phi[\nu^\Sigma(t), \nu] = \frac{\partial F}{\partial Y^\Sigma} \frac{\partial Y^\Sigma(x, t)}{\partial t}.$$

Функції $\Phi[\nu^\Sigma(t), \nu]$ та $u^*(x, t, \nu)$ будемо розглядати як суми двох додатків,
що залежать від різних наборів змінних

$$
\Phi[\nu^\Sigma(t), \nu] = \Phi[\nu^\Sigma(t)] + \overline{\Phi}_\nu[\nu^\Sigma(t), \nu],
$$

$$
u^*(x, t, \nu) = u^*(x, t) + \overline{u}_\nu^*(x, t, \nu).
$$

Відокремимо в системі (\ref{eq:major_non_lin_p1}) -- (\ref{eq:major_non_lin_p3}) лінійні члені від нелінійних
та перенесемо члени з
нелінійностями в праву частину системи і введемо малий параметр $\varepsilon$. Отримаємо систему

\begin{equation}
\label{eq:major_non_lin_epsi_p1}
a_1\frac{\partial\nu(x, t)}{\partial t} - a_2\frac{\partial^2\nu(x, t)}{\partial x^2} - \frac{K - 1}{x}a_2
\frac{\partial\nu(x, t)}{\partial x} - u_\Sigma(x, t) = \varepsilon H(\cdot), (x,t) \in S_0^T,
\end{equation}

\begin{equation}
\label{eq:major_non_lin_epsi_p2}
\begin{multlined}
a_2\frac{\partial\nu(x, t)}{\partial x} - \nu(1, t) = - \varepsilon h(\cdot), x = 1, t \in T_0,\\
\frac{\nu(x, t)}{\partial x} = 0, x = 0, t \in T_0,
\end{multlined}
\end{equation}


\begin{equation}
\label{eq:major_non_lin_epsi_p3}
\nu(x, t) = Y^{*0}(x), t = 0,
\end{equation}

де
%FIXME: change order of last elements
\begin{gather*}
H(\cdot) = \frac{\partial}{\partial x} \left[\Psi^*(\nu)\frac{\partial\nu(x, t)}{\partial} \right] -
\varphi^*(\nu)\frac{\partial\nu(x, t)}{\partial t} +
\frac{K - 1}{x}|Psi^*(\nu)\frac{\partial\nu(x, t)}{\partial x} + \\
+ [a_1 + \varphi^*(\nu)]\overline{\Phi}_\nu[\nu^\Sigma(y), \nu] +
\varphi(\nu)\Phi[\nu^\Sigma(t)] + \overline{u}_\nu^*(x, t, \nu), \\
h(\cdot) = \Psi(\nu)\frac{\partial\nu(x, t)}{\partial x},
x = 1, t \in T_0, u_\Sigma(x, t) = a_1\Phi[\nu^\Sigma(t)] + u^*(x, t).
\end{gather*}


При $\varepsilon = 1$ система (\ref{eq:major_non_lin_epsi_p1}) – (\ref{eq:major_non_lin_epsi_p3}) тотожна
початковій системі, з урахуванням вводу нової функції $\nu(x, t)$,
а також виразів для $C(y)$ та $\Lambda(y)$. При $\varepsilon = 0$ отримаємо «незбурену» задачу

\begin{gather*}
a_1\frac{\partial\nu(x, t)}{\partial t} - a_2\frac{\partial^2\nu(x, t)}{\partial x^2} - \frac{K - 1}{x}a_2
\frac{\partial\nu(x, t)}{\partial x} = u_\Sigma(x, t), \\
a_2\frac{\partial\nu(x, t)}{\partial x}  = \nu(1, t), x = 1, t \in T_0,\\
\frac{\nu(x, t)}{\partial x} = 0, x = 0, t \in T_0. \\
\nu(x, t) = Y^{*0}(x), t = 0.
\end{gather*}

Розв’язок нелінійної задачі (\ref{eq:major_non_lin_epsi_p1}) – (\ref{eq:major_non_lin_epsi_p3})
будемо шукати у вигляді ряду \cite{Mitropolskiy-Leksii}

$$
\nu(x,t, \varepsilon) = \nu_0(x, t) + \varepsilon\nu_1(x, t) + \varepsilon^2\nu_1(x, t) + \cdots
$$

Проведемо розклад функцій $H(\cdot)$ та $h(\cdot)$ в ряд Тейлора по степенях $\varepsilon$,
а також підставимо $\nu(x, t, \varepsilon)$ в систему
(\ref{eq:major_non_lin_epsi_p1}) – (\ref{eq:major_non_lin_epsi_p3}).
Групуючи члени при однакових степенях $\varepsilon$ отримаємо

\begin{gather*}
a_1\frac{\partial\nu_n(x, t)}{\partial t} - a_2\frac{\partial^2\nu_n(x, t)}{\partial x^2} -
\frac{K - 1}{x}a_2\frac{\partial\nu_n(x, t)}{\partial x} = \varepsilon H_n(\cdot), x\in S_0^T, \\
a_2\frac{\partial\nu_n(x, t)}{\partial x} - \nu_n(1, t) = -\varepsilon h_n(\cdot), x = 1, t\in T_0 \\
\frac{\partial\nu_n(x, t)}{\partial x} x = 0, t\in T_0 \\
\nu_n(x, t) = 0, t = 0, n = 1,2,\ldots
\end{gather*}

де $H_n(\cdot) = H_n(\nu_0(x, t), \nu_1(x, t),\ldots,\nu_{n-1}(x, t))$,
$h_n(\cdot) = h_n(\nu_0(x,t), \nu_1(x, t),$ $\ldots,\nu_{n-1}(x, t))$,  – розв’язок «незбуреної» задачі
при $\varepsilon = 0$. Послідовно розв’язуючи системи при $n=1,2,\ldots$ знайдемо $\nu_1(x, t), \nu_2(x, t),\ldots$

Згідно \cite{Furman-OVzaimosviazi} коефіцієнти апроксимації $a_1, a_2$ визначаються з виразів

\begin{gather*}
C_{21}(y) = \frac{1}{y_2(\cdot) - y_1(\cdot)}\int\limits_{y_1(\cdot)}^{y_2(\cdot)}C(y)dy = const,\\
\Lambda_{21}(y) = \frac{1}{y_2(\cdot) - y_1(\cdot)}\int\limits_{y_1(\cdot)}^{y_2(\cdot)}\Lambda(y)dy = const.
\end{gather*}

Підставляючи замість $C(y)$ та $\Lambda(y)$ відповідно $a_1\frac{\partial F[\cdot]}{\partial y}$ та
$a_2\frac{\partial F[\cdot]}{\partial y}$, знайдемо $a_1, a_2$ для інтервалу $[y_1(\cdot), y_2(\cdot)]$.
Отже, у нашому випадку
слабка нелінійність пропорційна апроксимованим значенням: $C(y)=a_1\frac{\partial F[\cdot]}{\partial y}$,
$\Lambda(y) = a_2\frac{\partial F[\cdot]}{\partial y}$. Функції
$\varphi(y) = C(y)\left(\frac{\partial F[\cdot]}{\partial y}\right)^{-1} - a_1$,
$\Psi(y) = \Lambda(y)\left(\frac{\partial F[\cdot]}{\partial y}\right)^{-1} - a_2$. Шукану функцію
$y(x, t)$
знайдемо підставляючи вираз $\nu(x, t) = F[Y^\Sigma(x,y), y]$ у наближений розв’язок
$\nu(x,t, \varepsilon) = \nu_0(x, t) + \varepsilon\nu_1(x, t) + \varepsilon^2\nu_1(x, t) + \cdots$ при
$\varepsilon = 1$.

\textbf{Висновки}. Метод малого параметра може бути успішно використаний для розв’язку задач зі суттєвими
нелінійностями, яки входять і в основне рівняння і в граничні умови. Це сприяє його подальшому розширенню на
задачі, що мають структурну неповноту початкової інформації.

\section{Використання функції Гріна для побудови множин псевдорозв’язків задач моделювання систем з розподіленими
параметрами} \label{sect1_3}
\subsection{Символічний метод Лур’є представлення розв’язків задач моделювання систем з розподіленими
параметрами} \label{sect1_3_1}

Одним із зручних методів розв’язку крайових задач (теплопровідності, теорії потенціалу) є
символічний метод Лур’є \cite{Lurie-Prostranstvennye}. Його переваги над іншими методами стають помітними
при розв’язку задач теорії пружності \cite{Lurie-KTeorii, Stoyan-ProRivniannia}. Суть метода полягає у
тому, щоб розглядати диференційні оператори у рівнянні як числа, якими можна оперувати як
алгебраїчними величинами. Розглянемо як приклад напівтривимірні рівняння осесиметричної задачі
динаміки пружних плит \cite{Skopetskiy-Matematychne}. Нехай пружній простір описується у циліндричній
системі координат $r,\theta, z$. Виділимо з нього шар площинами $z=\pm h$. Припустимо, що скручування
розглядуваного шару
відсутнє. Нормальні складові осесиметричних динамічних сил, що діють на граничні поверхні
шару позначимо через $Y_1^\Gamma(r, t)$ при $z = h$ та $Y_2^\Gamma(r, t)$ при $z=-h$, а дотичні –
через $Y_3^\Gamma(r, t)$ при $z=h$ та $Y_4^\Gamma(r, t)$ при $z=-h$.
$Y_i^\Gamma(r, t) \in C^n, i=1,2,3,4.$
Згідно зроблених припущень $Y_3\Gamma(r,t)$ і $Y_4\Gamma(r,t)$ мають тільки радіальні складові.
Якщо масові сили відсутні, то
вектор пружних динамічних зміщень точок такого шару визначається
радіальними $y_1(r, z, t)$ та нормальними $y_2(r, z, t)$ до
поверхонь $z=\pm h$ складовими \cite{Skopetskiy-Matematychne} і задовольняє рівнянням:

\begin{equation}
\label{eq:dinamika_plity}
\begin{multlined}
\frac{d^2}{dz^2}y_2(\cdot) + \frac{\mu}{\lambda + 2\mu}L_{22}^2(\partial_r^2, \partial_r, \partial_t^2)y_2(\cdot)+
\frac{\lambda + \mu}{\lambda + 2\mu}L(\partial_r)y_1(\cdot) = 0, \\
\frac{d^2}{dz^2}y_1(\cdot) + \frac{\lambda + 2\mu}{\mu}L_{11}^2(\partial_r^2, \partial_r, \partial_t^2)y_1(\cdot)+
\frac{\lambda + \mu}{\mu}\frac{\partial}{\partial r}\frac{d}{dz}y_2(\cdot) = 0, \\
\end{multlined}
\end{equation}


де $\partial_r$ – похідна по координаті $r$; $L(\partial_r) = \partial_r + \frac{1}{r}$ ;
$L_{mn}^2(\partial_r^2, \partial_r, \partial_t^2) = \Delta_n - \frac{1}{c_m^2}\partial_t^2$;
$m, n = 1, 2$; $c_1, c_2$ – швидкості поширення хвиль розши­рення та зсуву в
необмеженому пружному середовищі, $\partial_t$ – похідна по часу. Диференціальний оператор $\Delta_1$
пов’язаний з двовимірним оператором Лапласа $\Delta_2 = \partial_r^2 + \frac{1}{r}\partial_r$
співвідношенням $\Delta_1 \partial_r = \partial_r \Delta_2$. Граничні умови запишемо у вигляді

\begin{gather*}
\sigma_z(\cdot)\bigg|_{z=h} = Y_1^\Gamma(r, t), \sigma_z(\cdot)\bigg|_{z=-h} = Y_2^\Gamma(r, t), \\
\sigma_{rz}(\cdot)\bigg|_{z=h} = Y_3^\Gamma(r, t), \sigma_{rz}(\cdot)\bigg|_{z=-h} = Y_4^\Gamma(r, t), \\
\end{gather*}

де $\sigma_z(\cdot)$, $\sigma_{rz}(\cdot)$ – напруження нормальні та дотичні до поверхонь.

Введемо у розгляд деяку функцію $y(r,z,t)$, таку що $y_1(\cdot) = \frac{\partial y(\cdot)}{\partial r}$ .
Тоді перепишемо систему  (\ref{eq:dinamika_plity}) наступним чином

\begin{equation}
\label{eq:dinamika_plity_y}
\begin{multlined}
\frac{d^2}{dz^2}y_2(\cdot) + \frac{\mu}{\lambda + 2\mu}L_{22}^2(\partial_r^2, \partial_r, \partial_t^2)y_2(\cdot)+
\frac{\lambda + \mu}{\lambda + 2\mu} \Delta_2 \frac{d}{dz}y(\cdot) = 0, \\
\frac{d^2}{dz^2}y(\cdot) + \frac{\lambda + 2\mu}{\mu}L_{12}^2(\partial_r^2, \partial_r, \partial_t^2)y(\cdot)+
\frac{\lambda + \mu}{\mu}\frac{d}{dz}y_2(\cdot) = u(z, t), \\
\end{multlined}
\end{equation}

Використовуючи підстановку $y(\cdot) = \int y_1(r,z,t)dr + c(z, t)$, переконуємося що вибір функції
$u(z, t)$ не впливає на $y_1(r,z,t)$ та $y_2(r,z,t)$.
Отже, можемо прийняти $u(z, t)=0$. Тоді отримаємо систему однорідних диференційних рівнянь

\begin{equation}
\label{eq:dinamika_plity_y_u_0}
\begin{multlined}
\frac{d^2}{dz^2}y_2(\cdot) + \frac{\mu}{\lambda + 2\mu}L_{22}^2(\partial_r^2, \partial_r, \partial_t^2)y_2(\cdot)+
\frac{\lambda + \mu}{\lambda + 2\mu} \Delta_2 \frac{d}{dz}y(\cdot) = 0, \\
\frac{d^2}{dz^2}y(\cdot) + \frac{\lambda + 2\mu}{\mu}L_{12}^2(\partial_r^2, \partial_r, \partial_t^2)y(\cdot)+
\frac{\lambda + \mu}{\mu}\frac{d}{dz}y_2(\cdot) = 0. \\
\end{multlined}
\end{equation}



Будемо розглядати диференціальні оператори $\Delta_1$, $\Delta_2$ та $\partial_t^2$
у рівнянні (\ref{eq:dinamika_plity_y_u_0}) як алгебраїчні величини, а
самі рівняння як звичайні диференціальні рівняння з незалежною змінною $z$. Згідно \cite{Skopetskiy-Matematychne}
маємо загальний розв’язок системи (\ref{eq:dinamika_plity_y_u_0})

\begin{gather*}
y(\cdot) = \frac{\lambda + \mu}{\mu}[L_{12}(\cdot)\sin(zL_{12}(\cdot))A_1(\cdot) -
L_{12}(\cdot)\cos(zL_{12}(\cdot))A_2(\cdot) + \\
+L_{22}(\cdot)\sin(zL_{22}(\cdot))A_3(\cdot) -
L_{22}(\cdot)\cos(zL_{22}(\cdot))A_1(\cdot)], \\
y_2(\cdot) = \frac{\lambda + \mu}{\mu}[L_{12}^2\cos(zL_{12}(\cdot))A_1(\cdot) +
L_{12}^2\sin(zL_{12}(\cdot))A_2(\cdot) + \\
+\Delta_2\cos(zL_{22}(\cdot))A_3(\cdot) +
\Delta_2\sin(zL_{22}(\cdot))A_2(\cdot)],
\end{gather*}

де функції $A_1(\cdot)$, $A_2(\cdot)$, $A_3(\cdot)$, $A_4(\cdot)$ визначаються з крайових умов по
змінній $z$, символи $\sin(zL_{22}(\cdot))$ та $\cos(zL_{22}(\cdot))$ слід розуміти як степеневі ряди,
розкладені по $zL_{mn}(\cdot)$, причому, символу $L_{mn}(\cdot)$ необхідно повернути значення диференціального оператора для функції,
перед якою він написаний.

Використовуючи ідеї символічного методу, як показано в \cite{Skopetskiy-Matematychne} визначимо з
крайових умов невідомі функції $A_1(\cdot)$, $A_2(\cdot)$, $A_3(\cdot)$, $A_4(\cdot)$.


\textbf{Висновки}. Символічний метод Лур’є є зручним методом пошуку розв’язків лінійних задач теорії пружності. Ідею
оперування диференціальними операторами як алгебраїчними величинами можна використовувати для побудови функції Гріна,
що відповідає диференціальному рівнянню у частинних похідних у необмежених просторово часових областях. У подальших
розділах ми наведемо методику побудови функції Гріна, і таким чином,  розширимо символічний метод на інші класи задач.
Потім, ми перейдемо від моделей у нескінчених просторово-часових областях до моделей, що описують системи у
обмежених областях.

\subsection{Моделювання динаміки лінійних систем з розподіленими параметрами} \label{sect1_3_2}

Розглянемо методику моделювання систем з розподіленими параметрами
\cite{Blagoveshenskaya-Matematychne, Bogaenko-Avtomatyzatsiya, Skopetskiy-Matematychne,
Stoyan-DoPobudovy, Stoyan-Modeliuvannia}. Нехай стан системи
описується диференціальним рівнянням у частинних похідних в області $S_0^T$

\begin{equation}
\label{eq:stoyan_main_p1}
L_1(\partial_x^2, \partial_t) = L_2(\partial_x^2, \partial_t)u(s), s\in S_0^T,
\end{equation}

де

%FIXME: why N is equal in both, but m and n are not?
\begin{equation}
\label{eq:stoyan_main_p2}
L_1(\partial_x^2, \partial_t) = col(str(L_{1ij}(\partial_x^2, \partial_t))j=1,\ldots,n) i = 1,\ldots,N,
\end{equation}

\begin{equation}
\label{eq:stoyan_main_p3}
L_2(\partial_x^2, \partial_t) = col(str(L_{2ij}(\partial_x^2, \partial_t))j=1,\ldots,m) i = 1,\ldots,N,
\end{equation}

– відомі лінійні матричні диференціальні оператори з постійними коефіцієнтами при похідних, $s=(x,t)$,
$\partial_x^2 = \left(\frac{\partial^2}{\partial x_1^2}, \ldots, \frac{\partial^2}{\partial x_n^2} \right)$,
  – вектор
частинних похідних по просторових координатах $S_0^T = S_0\times T_0$,
$S_0 \subset S_\infty = (-\infty < x_i < \infty)$, $T_0 = [0; T]$, $y(s) = col[y_i(s)]\quad i = 1,\ldots,n$  –
вектор-функція, яка визначає стан системи, $u(s) = col[u_i(s)]\quad i=1,\ldots,m$ –
вектор-функція, що визначає зовнішньо-динамічні фактори, які діють на систему в області $S_0^T$.

Для повного опису системи задамо також початковий стан, та граничні умови

\begin{equation}
\label{eq:stoyan_edge_p1}
L_r^0(\partial_t)y(s) = Y_r^0(x), x \in S_0, t=0, r = 1,\ldots.R_0
\end{equation}

\begin{equation}
\label{eq:stoyan_edge_p2}
L_\rho^\Gamma(\partial_t)y(s) = Y_\rho^\Gamma(x), s \in \Gamma_0^T, \Gamma_0^T=\Gamma_0\times T_0, \rho = 1,\ldots.R_\Gamma
\end{equation}

\begin{equation}
\label{eq:stoyan_edge_p3}
L_r^0(\partial_t) = col(str(L_{rij}^0(\partial_t))i=1,\ldots,n)i=1,\ldots,N
\end{equation}

\begin{equation}
\label{eq:stoyan_edge_p4}
L_\rho^\Gamma(\partial_x) = col(str(L_{\rho ij}^\Gamma)i=1,\ldots,n)i=1,\ldots,N
\end{equation}

\begin{equation}
\label{eq:stoyan_edge_p5}
Y_r^0(x) = col(Y_{ri}^0(x))i=1,\ldots,N,\quad Y_{\rho}^{\Gamma}(s)=col(Y_{\rho i}^{\Gamma}(s))i=1,\ldots,N,
\end{equation}

де $\Gamma_0$ – границя $S_0$; $L_r^0(\partial_t)$, $L_{\rho}^\Gamma(\partial_x)$ – задані матричні
оператори, $Y_r^0(x)$, $Y_{\rho}^\Gamma(s)$  – задані векторні функції.

Існують дві класичні задачі:
\begin{itemize}
\item при відомих $u(s)$, $Y_r^0(s)$, $Y_{\rho}^\Gamma(s)$ знайти таке $y(s)$, що задовольняє
(або наближається з певною точністю) рівнянню (\ref{eq:stoyan_main_p1}) та крайовим
 умовам (\ref{eq:stoyan_edge_p1}) – (\ref{eq:stoyan_edge_p5}). (Пряма задача).
\item при відомому $y(s)$ знайти такі $u(s)$, $Y_r^0(s)$, $Y_{\rho}^\Gamma(s)$
які б привели систему (точно або наближено) до заданого стану. (Обернена задача).
\end{itemize}


Відомо багато методів розв’язку зазначених вище задач (або їх частинних випадків, коли замість матричних
диференціальних операторів
$L_r^0(\partial_t)$, $L_\rho^\Gamma(\partial_x)$, \\ $L_1(\partial_x^2, \partial_t)$, $L_2(\partial_x^2, \partial_t)$
задані звичайні диференціальні оператори): варіаційні методи
\cite{Ivanenko-Variatsyonnye, Kozdoba-Vychyslitelnaya, Rvachov-Metody, Elsgoltz-Differentsialnye}
 (Релея-Рітца, Бубнова-Гальоркіна, Лейбензона, Трефца, Канторовича, Біо та інші),
методи моментів, колокацій, методи скінчених різниць та скінчених елементів
\cite{Deyneka-Modeli, Segerlind-primenenie} метод суперелементів
\cite{Sergienko-Matematicheskoe}, метод регулярізації
\cite{Alifanov-ObratnieZadachiTeploobmena, Alifanov-PrimeneniePrinzipa, Alifanov-ReshenieNelineynoyObratnoy,
Tihonov-Metody, Tihonov-Nekorrektnye, Tihonov-Reguliariziruiuschie, Tihonov-Matematicheskoe, Tihonov-Chislennye}
 та метод квазіобернення і його модифікації
\cite{Lattes-MetodKvazi, Lions-NekotoryeMedody, Lions-Upravlenie, Lions-Optimalnoe}.

Однак, суттєві труднощі виникають, коли ми проводимо моделювання стану системи в умовах неповноти інформації,
тобто, порядок матричних операторів в (\ref{eq:stoyan_edge_p1}) – (\ref{eq:stoyan_edge_p2})
не узгоджений з порядком операторів у рівнянні (\ref{eq:stoyan_main_p1}),
відсутнє одне з рівнянь (\ref{eq:stoyan_edge_p1}) – (\ref{eq:stoyan_edge_p2}),
співвідношення (\ref{eq:stoyan_main_p1}) неадекватно описує динаміку системи.
Природно, що виникають задачі ідентифікації параметрів системи, моделювання зовнішньо динамічної обстановки,
що діє на систему та оптимізації розроблених моделей систем з розподіленими параметрами. Одним з варіантів
розв’язку таких задач є методика, запропонована В.А. Стояном
\cite{Stoyan-Modeliuvannia, Stoyan-OZadache, Stoyan-ObOpimizatsii}. Сутність її у наступному.
Розглядаючи диференціальні оператори
$\partial_x^2 = \left(\frac{\partial^2}{\partial x_1^2}, \cdots, \frac{\partial^2}{\partial x_n^2}\right)$
як алгебраїчні величини і відповідним чином оперуючи з ними,
побудуємо матричну функцію Гріна рівняння (\ref{eq:stoyan_main_p1})
для необмежених просторово часових областей. Маємо:

\begin{gather*}
G(x-x', t-t') = \frac{1}{(2\pi)^{n+1}}\int\limits_{S_\infty^\infty}Q(-\lambda^2, i\mu)\prod_{k=1}^n\exp
(i\lambda_k(x^k - x'_k))\exp(i\mu(t-t'))d\lambda d\mu,\\
Q(-\lambda^2, i\mu)=[L_1(-\lambda^2, i\mu)]^T[P_1(-\lambda^2, i\mu)]^{+}L_2(-\lambda^2, i\mu) = \\
=[P_2(-\lambda^2, i\mu)]^{+}[L_1(-\lambda^2, i\mu)]^T L_2(-\lambda^2, i\mu), \\
P_1(-\lambda^2, i\mu) = L_1(-\lambda^2, i\mu)[L_1(-\lambda^2, i\mu)]^T, \\
P_2(-\lambda^2, i\mu) = [L_1(-\lambda^2, i\mu)]^T L_1(-\lambda^2, i\mu),
\end{gather*}

де $P_1(-\lambda^2, i\mu)$ – матриця розміру $N\times N$, $P_2(-\lambda^2, i\mu)$ – матриця
розміру $m \times m$,  а $[P_1(-\lambda^2, i\mu)]^{+}$ та $[P_2(-\lambda^2, i\mu)]^{+}$ псевдообернені
\cite{Gantmaher-Teoriya} до них відповідно; $i$ – уявна одиниця.

За допомогою знайденої функції Гріна можемо записати розв’язок
$y_\infty(s)$ рівняння (\ref{eq:stoyan_edge_p1}) для необмежених
просторово-часових областей. Очевидно, що

\begin{equation}
\label{eq:stoyan_sol_infnty}
y_\infty(s)=\int\limits_{S_\infty^\infty}G(s-s')u(s')ds', s \in S_\infty^\infty, s=(x,t).
\end{equation}

Для ілюстрації цього факту  зазначимо, що будь-яку функцію можна представити у вигляді
$y(s)=\int\limits_{S_\infty^\infty}\delta(s-s')y(s')ds'$, де $\delta(s - s')$ – дельта-функція
Дірака. Запишемо $\delta(s - s')$ наступним чином
\cite{Korn-Spravochik}:

\begin{gather*}
\delta(s - s') = \delta(t - t')\prod_{k=1}^n\delta(x_k - x'_k), \\
\delta(t-t')=\frac{1}{2\pi}\int\limits_{-\infty}^{+\infty}\exp(i\mu(t-t'))d\mu, \\
\delta(x_k-x_k')=\frac{1}{2\pi}\int\limits_{-\infty}^{+\infty}\exp(i\lambda_k(x_k-x_k'))d\lambda_k. \\
\end{gather*}

Підставимо (\ref{eq:stoyan_sol_infnty}) в (\ref{eq:stoyan_main_p1}). Враховуючи, що
$(\partial_{x_k})^m\delta(x_k - x_k') = (i\lambda_k)^m\delta(x_k - x_k')$
отримаємо вірну тотожність.

Представимо розв’язок задачі (\ref{eq:stoyan_sol_infnty}) у вигляді
$y(s) = y_\infty(s) + y_0(s) + y_\Gamma(s)$, де функції
$y_0(s)$, $y(s)_\Gamma$  моделюють дію початково-крайових умов.
Згідно
\cite{Skopetskiy-Matematychne, Stoyan-Modeliuvannia} $y_0(s)$, $y_\Gamma(s)$ представимо у вигляді

\begin{gather*}
y_0(s) = \int\limits_{S_{\infty}^{\infty 0}}G(s-s')u_0(s')ds' \\
y_\Gamma(s) = \int\limits_{S_{\infty 0}^{T}}G(s-s')u_\Gamma(s')ds' \\
S_{\infty}^{\infty 0} = S_0 \times (T_\infty \ T_0), S_0 \subset S_\infty = (-\infty < x_i < +\infty) \\
S_{\infty 0}^{T}, T_0=[0; T_0], T_\infty(-\infty; T].
\end{gather*}

Зазначимо, що

\[
L_1(\partial_x^2, \partial_t)y(s) = L_1(\partial_x^2, \partial_t)\left\{
\begin{alignedat}{2}
u(s), & \quad s \in S_0^T \\
u(s), & \quad s \in S_{\infty}^{\infty 0} \\
u_\Gamma(s), & \quad s \in S_{\infty 0}^{T}
\end{alignedat}
\right.
\]

Отже, задача зводиться до пошуку невідомих $u_0(s)$, $u_\Gamma(s)$. Запишемо систему інтегральних рівнянь

\begin{equation}
\label{eq:stoyan_sol_integ}
\left\{
\begin{alignedat}{2}
&\int\limits_{S_\infty^{\infty 0}} L_r^0(\partial_t) G(s - s')u_0(s')ds' +
 \int\limits_{S_{\infty 0}^{T}} L_r^0(\partial_t) G(s - s')u_\Gamma(s')ds' = \\
& = Y_r^0(x) - \int\limits_{S_\infty^\infty}L_r^0(\partial_t)G(s-s')u(s')ds', x\in S_0,t=0,r=1,\ldots,R_0,\\
&\int\limits_{S_\infty^{\infty 0}} L_\rho^\Gamma(\partial_x) G(s - s')u_0(s')ds' +
 \int\limits_{S_{\infty 0}^{T}} L_\rho^\Gamma(\partial_x) G(s - s')u_\Gamma(s')ds' = \\
& = Y_\rho^\Gamma(s) - \int\limits_{S_\infty^\infty}L_\rho^\Gamma(\partial_x)G(s-s')u(s')ds', s\in \Gamma_0^T,\rho=1,\ldots,R_0,\\
\end{alignedat}
\right.
\end{equation}

де $\Gamma_0^T=\Gamma_0\times T_0$, $\Gamma_0$ – границя області $S_0$.

Для розв’язку системи інтегральних рівнянь (\ref{eq:stoyan_sol_integ}) виконаємо дискретизацію  змінних. Це можна зробити
трьома способами: дискретизуючи тільки $s$, дискретизуючи тільки $s'$ та дискретизуючи одночасно і $s$, і $s'$.
У першому випадку отримаємо

\begin{equation}
\label{eq:stoyan_sol_integ_s_first}
\left\{
\begin{alignedat}{2}
&\int\limits_{S_\infty^{\infty 0}} L_r^0(\partial_t) G(s_h - s')u_0(s')ds' +
 \int\limits_{S_{\infty 0}^{T}} L_r^0(\partial_t) G(s - s')u_\Gamma(s')ds' = \\
& = Y_r^0(x_h) - \int\limits_{S_\infty^\infty}L_r^0(\partial_t)G(s_h-s')u(s')ds', \\
 &x_h\in S_0,t=0,h=1,\ldots,H_0,r=1,\ldots,R_0,\\
&\int\limits_{S_\infty^{\infty 0}} L_\rho^\Gamma(\partial_x) G(s_w - s')u_0(s')ds' +
 \int\limits_{S_{\infty 0}^{T}} L_\rho^\Gamma(\partial_x) G(s_w - s')u_\Gamma(s')ds' = \\
& = Y_\rho^\Gamma(s_w) - \int\limits_{S_\infty^\infty}L_\rho^\Gamma(\partial_x)G(s_w-s')u(s')ds',\\
 &s_w\in \Gamma_0^T,w=1,\ldots,W_\Gamma,\rho=1,\ldots,R_0,\\
\end{alignedat}
\right.
\end{equation}

Запишемо систему (\ref{eq:stoyan_sol_integ_s_first}) у матричній формі \cite{Stoyan-Modeliuvannia}.

\begin{equation}
\label{eq:stoyan_sol_integ_s}
\int A(s')U(s')ds' = Y,
\end{equation}

\[
A(s') =
\left(
\begin{array}{cc}
G_{11}(s') & G_{12}(s') \\
G_{21}(s') & G_{22}(s') \\
\end{array}
\right),
\quad
U(s')=
\left(
\begin{array}{c}
u_{0}(s')  \\
u_{\Gamma}(s') \\
\end{array}
\right),
\quad
Y=
\left(
\begin{array}{c}
Y_0 \\
Y_\Gamma \\
\end{array}
\right),
\]
%FIXME: why G11 == G12 and G21 = G22 ?
\[
G_{11}(s')=col\left(col\left(L_r^0(\partial_t)G(s_h-s')\bigg|_{\substack{t=0\\x_h\in S_0}}\right)h=1,\ldots,H_0\right)r=1,\ldots,R_0,
\]
\[
G_{12}(s')=col\left(col\left(L_r^0(\partial_t)G(s_h-s')\bigg|_{\substack{t=0\\x_h\in S_0}}\right)h=1,\ldots,H_0\right)r=1,\ldots,R_0,
\]
\[
G_{21}(s')=col\left(col\left(L_\rho^\Gamma(\partial_x)G(s_w-s')\bigg|_{s_w\in\Gamma_0^T}\right)w=1,\ldots,W_\Gamma\right)\rho=1,\ldots,R_\Gamma,
\]
\[
G_{22}(s')=col\left(col\left(L_\rho^\Gamma(\partial_x)G(s_w-s')\bigg|_{s_w\in\Gamma_0^T}\right)w=1,\ldots,W_\Gamma\right)\rho=1,\ldots,R_\Gamma,
\]
\[
Y_0=col\left(col\left(Y_r^0(x_h) - \int\limits_{S_\infty^\infty}L_r^0G(s_h-s')\bigg|_{\substack{t=0\\x_h\in S_0}}\right)h=1,\ldots,H_0\right)r=1,\ldots,R_0
\]
\[
Y_\Gamma=col\left(col\left(Y_\rho^\Gamma(s_w) - \int\limits_{S_\infty^\infty}L_\rho^\Gamma G(s_w-s')\bigg|_{x_w\in \Gamma_0^T}\right)w=1,\ldots,W_\Gamma\right)\rho=1,\ldots,R_\Gamma
\]


Область інтегрування $\int A(s')U(s')ds'$ необхідно змінювати в залежності від добутку $A(s')U(s')ds$.
Наприклад, у випадку $G_{11}(s')u_0(s')$ маємо область інтегрування $S_{\infty}^{\infty 0}$.
Розв’язком задачі (\ref{eq:stoyan_sol_integ_s}) ми будемо називати таке $U(s')$, що

\[
U(s') = \arg\min_{U(s') \in \Omega_U}\int U(s)^2ds',\quad
\Omega_U = Arg \min_{U(s')}\left\|\int A(s')U(s')ds' - Y\right\|^2.
\]

Область інтегрування $\int U(s)^2ds'$  дорівнює $S_\infty^{\infty 0}$,
якщо підінтегральний добуток дорівнює $u_0(s')^2$, та $S_{\infty 0}^T$ – якщо $u_\Gamma(s')^2$.
Позначимо $P_1=\int A(s')[A(s')]^T ds'$, $P_2=\int [A(s')]^T A(s') ds'$.
Області інтегрування легко визначаються з (\ref{eq:stoyan_sol_integ_s}). Згідно
\cite{Stoyan-Obrachenie} у загальному випадку розв’язок задачі (\ref{eq:stoyan_sol_integ_s})
має вигляд

\[
U(s')\in\Omega_U = \left\{U(s'): U(s') = [A(s')]^T P_1^{+}Y + \nu(s') -
 [A(s')]^T P_1^{+}\int A(s')\nu(s')ds' \right\}
\]

тут $\nu(s')$ – довільна інтегрована в $S_\infty^{\infty 0}$ та $S_{\infty 0}^T$
вектор-функція розмірності 2, $P_1^+$ – псевдообернена
\cite{Stoyan-Modeliuvannia} до $P_1$ матрична функція.
Еквівалентне представлення $U(s')$ через $P_2$

\[
U(s')\in\Omega_U = \left\{U(s'): U(s') = P_2^{+} [A(s')]^T Y + \nu(s') -
  P_2^{+}[A(s')]^T\int A(s')\nu(s')ds' \right\}
\]

$P_2^+$ – псевдообернена до $P_2$ матрична функція.

Виконаємо дискретизацію тільки по змінній $s'$ і замінимо інтеграли інтегральними сумами.

\[
\sum_{\xi=1}^\Xi L_r^0(\partial_t)G(s-s'_\xi) u_0(s'_\xi)\bigg|_{s'_\xi\in S_{\infty}^{\infty 0}}\Delta s'_\xi +
\sum_{\varphi=1}^\Phi L_r^0(\partial_t)G(s-s'_\varphi) u_\Gamma(s'_\varphi)\bigg|_{s'_\varphi \in S_{\infty 0}^{T}}\Delta s'_\varphi =
\]
\[
=Y_r^0(x)-\int\limits_{S_\infty^\infty}L_r^0(\partial_t)G(s-s')u(s')ds',\quad x\in S_0, t=0, r=1,\ldots,R_0,
\]
\[
\sum_{\xi=1}^\Xi L_\rho^\Gamma(\partial_x)G(s-s'_\xi) u_0(s'_\xi)\bigg|_{s'_\xi\in S_{\infty}^{\infty 0}}\Delta s'_\xi +
\sum_{\varphi=1}^\Phi L_\rho^\Gamma(\partial_x)G(s-s'_\varphi) u_\Gamma(s'_\varphi)\bigg|_{s'_\varphi \in S_{\infty 0}^{T}}\Delta s'_\varphi =
\]
\[
=Y_\rho^\Gamma(s)-\int\limits_{S_\infty^\infty}L_\rho^\Gamma(\partial_x)G(s-s')u(s')ds',\quad s\in \Gamma_0^T, \rho=1,\ldots,R_\Gamma.
\]


Запишемо систему функціональних рівнянь

\begin{equation}
\label{eq:stoyan_sol_func_s_dash}
B(s)U=Y(s), s\in S_0^T,
\end{equation}

де

\[
B(s) =
\left(
\begin{array}{cc}
G_{11}(s) & G_{12}(s) \\
G_{21}(s) & G_{22}(s) \\
\end{array}
\right),
\quad
U=
\left(
\begin{array}{c}
U_{0}  \\
U_{\Gamma} \\
\end{array}
\right),
\quad
Y(s)=
\left(
\begin{array}{c}
Y_0(s) \\
Y_\Gamma(s) \\
\end{array}
\right),
\]

\[
G_{11}(s)=col\left(str\left(L_r^0(\partial_t)G(s-s'_\xi)\bigg|_{s'_\xi\in S_\infty^{\infty 0}} \Delta s'_\xi \right)\xi=1,\ldots,\Xi\right)r=1,\ldots,R_0,
\]
\[
G_{12}(s)=col\left(str\left(L_r^0(\partial_t)G(s-s'_\varphi)\bigg|_{s'_\varphi\in S_{\infty 0}^{T}} \Delta s'_\varphi \right)\varphi=1,\ldots,\Phi\right)r=1,\ldots,R_0,
\]
\[
G_{21}(s)=col\left(str\left(L_\rho^\Gamma(\partial_x)G(s-s'_\xi)\bigg|_{s'_\xi\in S_\infty^{\infty 0}} \Delta s'_\xi \right)\xi=1,\ldots,\Xi\right)\rho=1,\ldots,R_0,
\]
\[
G_{22}(s)=col\left(str\left(L_\rho^\Gamma(\partial_t)G(s-s'_\varphi)\bigg|_{s'_\varphi\in S_{\infty 0}^{T}} \Delta s'_\varphi \right)\varphi=1,\ldots,\Phi\right)\rho=1,\ldots,R_0,
\]
\[
U_0=col\left(u_0(s'_\xi)\bigg|_{s'_\xi\in S_\infty^{\infty 0}}\right)\xi=1,\ldots,\Xi,\quad
U_\Gamma=col\left(u_\Gamma(s'_\varphi)\bigg|_{s'_\varphi\in S_{\infty 0}^{T}}\right)\varphi=1,\ldots,\Phi,
\]
\[
Y_0(s) = col\left(Y_r^0(x) - \int\limits_{S_\infty^\infty}L_r^0(\partial_t)G(s-s')u(s')ds' \right)r=1,\ldots,R_0,
\]
\[
Y_\Gamma(s) = col\left(Y_\rho^\Gamma(s) - \int\limits_{S_\infty^\infty}L_r^0(\partial_x)G(s-s')u(s')ds' \right)\rho=1,\ldots,R_\Gamma.
\]


Згідно \cite{Skopetskiy-Matematychne} розв’язком задачі (\ref{eq:stoyan_sol_func_s_dash}) ми будемо називати таке $U$, що

\[
U = \arg\min_{U \in \Omega_U}\left\|U\right\|^2,\quad
\Omega_U = Arg \min_{U}\int\limits_{S_0^T}\left\|B(s)U - Y(s)\right\|^2ds.
\]


Позначимо $P_1=\int\limits_{S_0^T}B(s)[B(s)]^Tds$, $P_2=\int\limits_{S_0^T}[B(s)]^TB(s)ds$.
Загальний розв’язок системи (\ref{eq:stoyan_sol_func_s_dash}) має вигляд

\[
U\in\Omega_U=\left\{U:U = \int\limits_{S_0^T}[B(s)]^T P_1^+Y(s)ds +\nu-
\int\limits_{S_0^T}[B(s)]^TP_1^+B(s)\nu ds\right\}
\]

або

\[
U\in\Omega_U=\left\{U:U = \int\limits_{S_0^T} P_2^+[B(s)]^TY(s)ds +\nu-
\int\limits_{S_0^T}P_2^+[B(s)]^TB(s)\nu ds\right\},
\]

де $\nu$ – довільний вектор розмірності $\Xi+\Phi$; $P_1^+$, $P_2^+$ – псевдообернені матричні
функції відповідно до $P_1$ та $P_2$. Операція
псевдообернення виконується згідно \cite{Skopetskiy-Matematychne}.

Розглянемо третій випадок, коли дискретизація виконується одночасно по змінних $s$ та $s'$. Отримаємо систему
лінійних алгебраїчних рівнянь

\[
\sum_{\xi=1}^\Xi L_r^0(\partial_t)G(s_h-s'_\xi) u_0(s'_\xi)\bigg|_{s'_\xi\in S_{\infty}^{\infty 0}}\Delta s'_\xi +
\sum_{\varphi=1}^\Phi L_r^0(\partial_t)G(s_h-s'_\varphi) u_\Gamma(s'_\varphi)\bigg|_{s'_\varphi \in S_{\infty 0}^{T}}\Delta s'_\varphi =
\]
\[
=Y_r^0(x_h)-\int\limits_{S_\infty^\infty}L_r^0(\partial_t)G(s_h-s')u(s')ds',\quad x\in S_0, t=0, h=1,\ldots,H_0, r=1,\ldots,R_0,
\]
\[
\sum_{\xi=1}^\Xi L_\rho^\Gamma(\partial_x)G(s_w-s'_\xi) u_0(s'_\xi)\bigg|_{s'_\xi\in S_{\infty}^{\infty 0}}\Delta s'_\xi +
\sum_{\varphi=1}^\Phi L_\rho^\Gamma(\partial_x)G(s_w-s'_\varphi) u_\Gamma(s'_\varphi)\bigg|_{s'_\varphi \in S_{\infty 0}^{T}}\Delta s'_\varphi =
\]
\[
=Y_\rho^\Gamma(s_w)-\int\limits_{S_\infty^\infty}L_\rho^\Gamma(\partial_x)G(s_w-s')u(s')ds',\quad s\in \Gamma_0^T, w=1,\ldots,W_\Gamma, \rho=1,\ldots,R_\Gamma.
\]


Запишемо систему у матричному вигляді.

\begin{equation}
\label{eq:stoyan_sol_matrix_s_s_dash}
CU=Y,
\end{equation}

де

\[
C =
\left(
\begin{array}{cc}
G_{11} & G_{12} \\
G_{21} & G_{22} \\
\end{array}
\right),
\quad
U=
\left(
\begin{array}{c}
U_{0}  \\
U_{\Gamma} \\
\end{array}
\right),
\quad
Y=
\left(
\begin{array}{c}
Y_0 \\
Y_\Gamma \\
\end{array}
\right),
\]

\[
G_{11}(s)=col\left(col\left(str\left(L_r^0(\partial_t)G(s_h-s'_\xi)\bigg|_{\substack{x_h\in S_0,t=0 \\ s'_\xi\in S_\infty^{\infty 0}}} \Delta s'_\xi \right)\scriptstyle{\xi=\overline{1,\Xi}}\right)\scriptstyle{h=\overline{1,H_0}}\right)\scriptstyle{r=\overline{1,R_0}},
\]
\[
G_{12}(s)=col\left(col\left(str\left(L_r^0(\partial_t)G(s_h-s'_\varphi)\bigg|_{\substack{x_h\in S_0,t=0 \\ s'_\varphi\in S_{\infty 0}^{T}}} \Delta s'_\varphi \right)\scriptstyle{\varphi=\overline{1,\Phi}}\right)\scriptstyle{h=\overline{1,H_0}} \right)\scriptstyle{r=\overline{1,R_0}},
\]
\[
G_{21}(s)=col\left(col\left(str\left(L_\rho^\Gamma(\partial_x)G(s_w-s'_\xi)\bigg|_{\substack{s_w\in\Gamma_0^T \\ s'_\xi\in S_\infty^{\infty 0}}} \Delta s'_\xi \right)\scriptstyle{\xi=\overline{1,\Xi}}\right)\scriptstyle{w=\overline{1,W_\Gamma}}\right)\scriptstyle{\rho=\overline{1,R_0}},
\]
\[
G_{22}(s)=col\left(col\left(str\left(L_\rho^\Gamma(\partial_t)G(s_w-s'_\varphi)\bigg|_{\substack{s_w\in\Gamma_0^T \\ s'_\varphi\in S_{\infty 0}^{T}}} \Delta s'_\varphi \right)\scriptstyle{\varphi=\overline{1,\Phi}}\right)\scriptstyle{w=\overline{1,W_\Gamma}}\right)\scriptstyle{\rho=\overline{1,R_0}},
\]
\[
U_0=col\left(u_0(s'_\xi)\bigg|_{s'_\xi\in S_\infty^{\infty 0}}\right)\scriptstyle{\xi=1,\ldots,\Xi},\quad
U_\Gamma=col\left(u_\Gamma(s'_\varphi)\bigg|_{s'_\varphi\in S_{\infty 0}^{T}}\right)\scriptstyle{\varphi=1,\ldots,\Phi},
\]
\[
Y_0(s) = col\left(col\left(Y_r^0(x_h) - \int\limits_{S_\infty^\infty}L_r^0(\partial_t)G(s_h-s')\bigg|_{\substack{t=0 \\ x_h\in S_0}}u(s')ds' \right)\scriptstyle{h=\overline{1,H_0}} \right)\scriptstyle{r=\overline{1,R_0}},
\]
\[
Y_\Gamma(s) = col\left(col\left(Y_\rho^\Gamma(s_w) - \int\limits_{S_\infty^\infty}L_r^0(\partial_x)G(s_w-s')\bigg|_{s_w\in \Gamma_0^T}u(s')ds' \right)\scriptstyle{w=\overline{1,W_\Gamma}} \right)\scriptstyle{\rho=\overline{1,R_\Gamma}}.
\]


Згідно \cite{Blagoveshenskaya-Matematychne, Stoyan-Modeliuvannia} розв’язком системи
(\ref{eq:stoyan_sol_matrix_s_s_dash}) буде таке $U$,
що належить до множини $\Omega_U$. Отже,

\[
U\in\Omega_U = \left\{U:U=C^TP_1^+Y+I\nu-C^TP_1^+C\nu\right\}
\]

або

\[
U\in\Omega_U = \left\{U:U=P_2^+C^TY+I\nu-P_2^+C^TC\nu\right\},
\]

де $P_1=CC^T$, $P_2=C^TC$, $\nu$ – довільний вектор розмірності $\Xi R_0 + \Phi R_\Gamma$;
$P_1^+$, $P_2^+$  – матриці псевдо-обернені \cite{Gantmaher-Teoriya} до $P_1$ та $P_2^+$
відповідно, $I$ – одинична матриця.

\textbf{Висновки}. Функція Гріна є зручним інструментом для моделювання розв’язків лінійних систем з розподіленими
параметрами. Методика оперування диференційними операторами як алгебраїчними величинами дозволяє стандартним
єдиним способом знайти розв’язки широкого класу лінійних диференціальних рівнянню Крім того описана вище методика
може бути використана для як для прямих задач так і для обернених. Вона також дозволяє проводити моделювання в
умовах структурної неповноти інформації.

\section{Предмет та методика наукового дослідження} \label{sect1_4}

Показано, що існує велика кількість методів розв’язку як прямих так і обернених задач теплопровідності. Існує
можливість підібрати той чи інший метод  і провести моделювання широкого класу задач.  Але не існує загальної
методики моделювання прямих та обернених нелінійних задач хоча б для певного вузького класу нелінійностей. Також
не існує загальної методики отримання оцінок точностей для знайдених  розв’язків.

Розглянуто основні концепції метода малого параметра. Встановлено, що сфера застосування метода є досить широкою.
Параметр може бути присутнім в граничних умовах, початкових умовах та у диференціальному рівнянні. Збурення може
бути проведено по координатах часу і простору та у диференціальному операторі у якості безрозмірних коефіцієнтів.
Малий параметр може бути введений штучно. Все це додатково розширює область застосування метода.

Показано, що метод малого параметра може бути успішно використаний для розв’язку задач зі суттєвими нелінійностями,
яки входять і в основне рівняння і в граничні умови. Це сприяє його подальшому розширенню на нові класи задачі.

Зазначено, що символічний метод Лур’є є зручним методом пошуку розв’язків лінійних задач теорії пружності.
Ідея оперування диференціальними операторами як алгебраїчними величинами є основою для методики побудови функції
Гріна, що відповідає диференціальному рівнянню у частинних похідних у необмежених просторово часових областях.

Показано, що функція Гріна може бути використана для пошуку розв’язків задач моделювання лінійних систем в
умовах структурної неповноти інформації.

Сказане вище дозволяє зробити висновок про доцільність об’єднання  метода малого параметра та методики
моделювання розв’язків систем за допомогою функції Гріна з метою знаходження розв’язків нового класу задач:
нелінійних прямих (та обернених) задач з неповною початковою інформацією.

Нова методика дозволить однотипно розв’язувати як прямі так і обернені (задача спостереження) задачі.
