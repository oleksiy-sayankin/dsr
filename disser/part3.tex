\chapter{Математичне моделювання обернених задач динаміки нелінійних систем та оптимізація
моделей динаміки нелінійних систем} \label{chapt3}

В розділі нами розглянуто питання побудови середньо квадратичних наближень (якщо вони існують) до розв’язків
нелінійних в малому обернених задач заданих сумою лінійного диференціального оператора та нелінійного збурення в
умовах неповноти інформації. Нелінійне збурення  представлене поліномом зі сталими коефіцієнтами від шуканої
функції та її часткових похідних по координатах.

Серед класу обернених задач нами розглянуто задачі відновлення початкового стану при відомих крайових умовах та
керуючий функції та задачі відновлення крайових умов при відомому початковому стану  та керуючий функції.

Наближення до розв’язків оберненої нелінійної задачі нами побудовано на базі методики розв’язання квазілінійних
задач (метод збурень) описаної у першому розділі даної дисертаційної роботи.

Середньо квадратичні наближення (якщо вони існують) до розв’язків допоміжних обернених лінійних задач нами
побудовано на основі методики середньоквадратичного обернення систем функціональних та інтегральних співвідношень
також описаної у першому розділі роботи.

Нами також розглянута задача оптимального розміщення  точок дискретизації спостережень за зовнішньо-динамічними
характеристиками системи та задача оптимального розміщення  точок дискретизації моделюючих факторів системи.

Розв’язок задач оптимізації нами виконано градієнтним методом.

\section{Математичне моделювання обернених задач динаміки нелінійних систем} \label{sect3_1}

\subsection{Задача відновлення початкового стану для нелінійних систем при відомих крайових
умовах та керуючий функції} \label{sect3_1_1}


Розглянемо систему, стан якої описується наступним диференціальним рівнянням, граничною умовою та спостереженнями
за системою. Зазначимо, що початковий стан $Y^{0}$ системи
(\ref{eq:small_parameter_common_system_with_no_init_condition}) невідомий.

\begin{equation}
    \label{eq:small_parameter_common_system_with_no_init_condition}
    \left\{
        \begin{alignedat}{2}
            L(\partial_x^2, \partial_t)y(s) + \varepsilon P(y(s), \partial_x y(s), \partial_x^2 y(s)) = u(s), s\in S_{0}^T, \\
            L^{\Gamma}(\partial^{2}_{x})y(s) = Y^{\Gamma}(s),s\in\bigcup\limits_{\rho=1}^{R_{\Gamma}} \Gamma_{0\rho}\times T_{0}, \\
            L^{c}(\partial^{2}_{x})y(s) = Y^{c}(s), s\in {S'}_{0}^T,
        \end{alignedat}
    \right.
\end{equation}
де $s=(x,t)$, $x=(x_{1},\dots,x_{n})$, $S_{0}^{T}=S_{0}\times T_{0}$, ${S'}_{0}^{T}={S'}_{0}\times {T'}_{0}$,
$\partial_{x}=\left(\frac{\partial}{\partial x_{1}}, \dots, \frac{\partial}{\partial x_{n}} \right)$,
$\partial^{2}_{x}=\left(\frac{\partial^{2}}{\partial x^{2}_{1}}, \dots, \frac{\partial^{2}}{\partial x^{2}_{n}} \right)$,
$L(\partial_x^2, \partial_t)$, $L^{\Gamma}(\partial_{x})$ -- лінійні диференціальні оператори з постійними коефіцієнтами при похідних,
$
L(\partial_x^2, \partial_t)=\sum\limits_{ \substack{i,j=1\\ i\neq j}}\left(l_{1i} \frac{\partial^2}{\partial x^2_i} +
l_{2i}\frac{\partial^2}{\partial x_{1i}
    \partial x_{1j}} + l_{3i} \frac{\partial}{\partial x_i} \right) + l_4 \frac{\partial}{\partial t}
$,
$
L^{\Gamma}(\partial_x^2)=\sum\limits_{ \substack{i,j=1\\ i\neq j}}\left(l^{\Gamma}_{1i} \frac{\partial^2}{\partial x^2_i} +
l^{\Gamma}_{2i}\frac{\partial^2}{\partial x_{1i}
    \partial x_{1j}} + l^{\Gamma}_{3i} \frac{\partial}{\partial x_i} \right)
$,
$l_{1i}$, $l_{2i}$, $l_{3i}$, $l_{4}\in\mathbb{R}$,
$l^{\Gamma}_{1i}$, $l^{\Gamma}_{2i}$, $l^{\Gamma}_{3i}$,

$P(y(s), \partial_x y(s), \partial_x^2 y(s))$ -- многочлен своїх аргументів з постійними
коефіцієнтами при похідних, $\varepsilon$ – малий параметр, $\varepsilon\in [0; 0.01]$,
$Y^{\Gamma}(x)\in\mathbb{C}^n$ -- задана функція, визначена в
$\bigcup\limits_{\rho=1}^{R_{\Gamma}}\Gamma_{0\rho}\times T_{0}$;
$\Gamma_{0\rho}\subseteq \Gamma_{0}$; $\Gamma_{0\rho}\cap\Gamma_{0\sigma} = \varnothing$;
$\rho\neq\sigma$; $\rho,\sigma = 1,\dots,R_{\Gamma}$, $R_{\Gamma}\in\mathbb{N}$; $u(s)\in\mathbb{C}^n$ --
задана функція, визначена в $S_{0}^{T}$, $S\subset S_{\infty}=\mathbb{R}^{n}$, $Y^{c}(s)\in\mathbb{C}^n$ -- задана функція (спостереження за системою), визначена в $S_{0}^{T}$,
$S_{0}$ -- зв'язна замкнена множина; ${S'}_{0}\subset S_{0}$; $\Gamma_{0}$ -- границя області $S_{0}$,
$k,n\in\mathbb{N}$, $T_{0}=[0;T]$, ${T'}_{0}\subset T_{0}$, $T\in\mathbb{R}_{+}$.

Для пошуку $Y^{0}(x)$, $x\in S_{0}$, $t=0$ знайдемо $y(s)$.
Тоді  $Y^{0}(x)=y(x,0)$.
Нехай

\[
    y(s)=y^{(0)}(s) + \varepsilon y^{(1)}(s) + \varepsilon^2 y^{(2)}(s) + \dots
\]

де $y^{(0)}(s), y^{(1)}(s), y^{(2)}(s),\dots\in\mathbb{C}^{n}$ -- довільні функції, визначені в $S_{\infty}^{\infty}$.

Підставимо представлення $y(s)$ у вигляді ряду в початкове рівняння. Використовуючи розглянуту вище методику, запишемо систему рівнянь




\textbf{Висновки}. Розглянуто задачу відновлення початкового стану нелінійної системи при відомих спостереженнях за
системою і функцією зовнішньо динамічних збурень. Розроблена методика розв’язання задачі в умовах неповноти
інформації: гранична умова відома лише на певних областях границі. Знайдені множини моделюючих функцій,
які дозволяють відновити початковий стан системи.

\subsection{Задача відновлення крайових умов для нелінійних систем при відомому початковому
стану та керуючий функції} \label{sect3_1_2}

\textbf{Висновки}. Розглянуто нелінійну систему для якої початкові умови відомі не на всій області пошуку функції,
яка моделює стан системи, а лише в деяких підобластях. Розроблена методика зведення нелінійної системи з малим
параметром до сукупності лінійних систем. Для кожної з лінійних систем знайдена функція Гріна, що відповідає
випадку нескінчених просторово-часових областей. Розроблена методика і проведено моделювання дії початкової
умови з урахуванням спостережень з системою. Побудовані множини функцій, за допомогою яких відновлені крайові
умови, що приводять до спостережуваного стану системи. Виконана оцінка точності отриманих результатів.

\section{Оптимізація моделей динаміки нелінійних систем} \label{sect3_2}

\subsection{Задача оптимального розміщення  точок дискретизації спостережень за зовнішньо-динамічними
характеристиками системи} \label{sect3_2_1}

\textbf{Висновки}. Розглянута задача оптимального розміщення  точок дискретизації спостережень за
зовнішньо-динамічними характеристиками системи. Знайдений алгоритм пошуку оптимального вектору точок.
Знайдені вирази для диференціювання початкової та псевдооберненої матриці.

\subsection{Задача оптимального розміщення  точок дискретизації моделюючих факторів системи} \label{sect3_2_2}

\textbf{Висновки}. Розглянута задача оптимального розміщення точок дискретизації моделюючих факторів системи.
Знайдено алгоритм пошуку розв’язку цієї задачі. Визначені вирази для диференціювання початкової та
псевдооберненої матриці, яка є результатом дискретизації інтегральних рівнянь, що моделюють початкову крайову задачу.

\section*{Висновки до третього розділу}
\addcontentsline{toc}{section}{Висновки до третього розділу}

Розглянуто задачу відновлення початкового стану нелінійної системи при відомих спостереженнях за системою і функцією
зовнішньо динамічних збурень. Розроблена методика розв’язання задачі в умовах неповноти інформації: гранична умова
відома лише на певних областях границі. Знайдені множини моделюючих функцій, які дозволяють відновити початковий
стан системи.

Розглянуто нелінійну систему для якої початкові умови відомі не на всій області пошуку функції, яка моделює стан
системи, а лише в деяких підобластях. Розроблена методика зведення нелінійної системи з малим параметром до
сукупності лінійних систем. Для кожної з лінійних систем знайдена функція Гріна, що відповідає випадку нескінчених
просторово-часових областей. Розроблена методика і проведено моделювання дії початкової умови з урахуванням
спостережень з системою. Побудовані множини функцій, за допомогою яких відновлені крайові умови, що приводять
до спостережуваного стану системи. Виконана оцінка похибки моделювання отриманих результатів.

Розглянута задача оптимального розміщення  точок дискретизації спостережень за зовнішньо-динамічними характеристиками
системи. Знайдений алгоритм пошуку оптимального вектору точок. Знайдені вирази для диференціювання початкової та
псевдооберненої матриці.

Розглянута задача оптимального розміщення  точок дискретизації моделюючих факторів системи. Знайдено алгоритм
пошуку розв’язку цієї задачі. Визначені вирази для диференціювання початкової та псевдооберненої матриці, яка є
результатом дискретизації інтегральних рівнянь, що моделюють початкову крайову задачу.
