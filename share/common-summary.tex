\begin{enumerate}
  \item Розглянуто нелінійні системи, які функціонують у нескінченій області, в обмеженій часовій області, обмеженій
  просторовій області та скінченій просторо-часовій області.Отримано наближення нелінійної системи через нескінчену
  кількість лінійних систем.
  \item Показано, що функція Гріна може бути використана для пошуку розв’язків задач моделювання лінійних систем в
  умовах структурної неповноти інформації.
  \item Запропонована методика моделювання дії початково-крайових умов за допомогою фіктивних функцій, які визначені за
  межами області пошуку розв’язків.
  \item Побудовані множини моделюючих функцій для різних видів дискретизації початково-крайових умов, а також для
  випадку  обмеженій тільки в часовій області та обмеженій тільки в просторовій області.
  \item Для кожної з лінійних систем, які моделюють стан нелінійної системи,  побудовано функції Гріна.
  \item Проведено моделювання розв’язків лінійних систем та отримано розв’язок початкової системи у вигляді ряду.
  Знайдено похибку моделювання для нелінійних систем, що функціонують у скінченій просторово-часовій області.
  \item Розглянуто задачу відновлення початкового стану нелінійної системи при відомих спостереженнях за системою і
  функцією зовнішньо динамічних збурень. Розроблена методика розв’язання задачі в умовах неповноти інформації:
  гранична умова відома лише на певних областях границі. Знайдені множини моделюючих функцій, які дозволяють
  відновити початковий стан системи.
  \item Розглянуто нелінійну систему для якої початкові умови відомі не на всій області пошуку функції, яка моделює
  стан системи, а лише в деяких підобластях. Розроблена методика зведення нелінійної системи з малим параметром до
  сукупності лінійних систем. Для кожної з лінійних систем знайдена функція Гріна, що відповідає випадку нескінчених
  просторово-часових областей. Розроблена методика і проведено моделювання дії початкової умови з урахуванням
  спостережень з системою. Побудовані множини функцій, за допомогою яких відновлені крайові умови, що приводять до
  спостережуваного стану системи. Виконана оцінка точності отриманих результатів.
  \item Розглянута задача оптимального розміщення  точок дискретизації моделюючих факторів системита задача
  оптимального розміщення  точок дискретизації спостережень за зовнішньо-динамічними характеристиками системи.
  Знайдено алгоритм пошуку розв’язку цих задачі. Визначені вирази для диференціювання початкової та псевдооберненої
  матриці, яка є результатом дискретизації інтегральних рівнянь, що моделюють початкову крайову задачу.
  \item Проведено моделювання динаміки нелінійної системи в необмежених просторово-часових областях.
  Встановлена абсолютна величина максимальної похибки моделювання. Розглянутанелінійна задача поширення тепла в
  обмеженій просторово-часовій області в умовах структурної неповноти інформації. Проведено математичне моделювання
  розв’язків та виконана оцінка похибки у порівнянні із задачею при повністю заданих початково-крайових умовах.
\end{enumerate}